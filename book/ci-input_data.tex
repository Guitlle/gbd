\section{DisMod III Input Data}

The input data files for DisMod III are stored as Javascript Object
Notation (JSON) files.  This is conveniently readable by computers,
and moderately readable by humans, and provides a lot of flexibility.
It can prove unwieldly for very large datasets, but so far it has
worked fine.  Support of compressed JSON files is planned for the day
when datasets become too large to store as plain text files.

The format of the DisMod III input JSON is a dictionary with $5$ entries:
\begin{verbatim}
{ "input_data" : input_data_csv (required),
  "parameters" : param_list (required),
  "output_template" : output_template_csv (required),
  "areas_hierarchy" : [[parent_area, child_area], ...] (required), list of pairs of areas,
  "areas_to_fit" : [area_1, area_2, ...] (required), list of areas }
\end{verbatim}

Each of these entries is described in more detail now.

\subsection{Input Data}
The input data comma-separated value (CSV) is a text file of tabular
data specifying all of the data that will go into the likelihood of
the model.

The CSV begins with a row of column headers, and each row of the CSV after the column headers takes the following form:
\begin{verbatim}
      "data_type" : str (required), one of the following types:
                      "i",
                      "p",
                      "r",
                      "f",
                      "rr",
                      "smr",
                      "m_with",
                      "X",
                      "csmr",
                      "pf"

      "value" : float (required), parameter value limits
                "i" >= 0,
                "p" [0, 1],
                "r" >= 0,
                "f" >=0,
                "rr" >= 0,
                "smr" >= 0,
                "m_with" >= 0,
                "X" >= 0,
                "csmr"  >= 0,
                "pf" >= 0

      "area" : str (required), a geographic area defined in the area table,

      "sex" : str (required), one of "male", "female", "total",

      "age_start" : int[0, 150], <= age_end (required),

      "age_end" : int[0, 150], >= age_start (required),

      "year_start" : int[1900, 2050], <= year_end (required),

      "year_end" : int[1900, 2050], >= year_start (required),

      "standard_error" : float > 0 (optional*),

      "effective_sample_size" : int > 0 (optional*),

      "lower_ci" : float >= 0 <= Parameter Value (optional*),
      
      "upper_ci" : float > Parameter Value (optional*),

    (* se is required for X, ci is required for rr, smr, ess is required for all others)

      "age_weights" : [ float, float, ... ] (required), length equals age_end - age_start + 1,
                      default/missing assume to be [ 1, ... ],

      additional keys, with corresponding values for all study-level covariates, and all country-level   
      covariates merged for this data_type, this region, this sex, this year_start and this year_end
      starting with x_ if they are included in mean and u_ if the are included in dispersion
      for example:
      "x_time" : float
      "u_biased" : [0,1]
\end{verbatim}


\subsection{Parameters}
\begin{verbatim}
    param_list = {
      "p" : param_dict (required), see below,
      "i" : param_dict (required), see below,
      "r" : param_dict (required), see below,
      "f" : param_dict (required), see below,
      "X" : param_dict (required), see below,
      "rr" : param_dict (required) see below,
    }

    param_dict = {
      "smoothness" : {
        "amount" : str (required), one of "Slightly", "No Prior", "Moderately", "Very"], default "Slightly",

        "age_start" : int[0, 100], <= "age_end" (required), default 0,

        "age_end" : int[0, 100], >= age_start (required), default 100
      },

      "heterogeneity" : str (required), one of "Slightly", "Moderately", "Very", "Unusable", default "Slightly",

      "level_value" : {
        "value" : float >= level_bounds["lower"], <= level_bounds["upper"] (required), default 0,

        "age_before" : int[0, 100], <= age_after (required), default 0,

        "age_after" : int[0, 100], >= age_before (required), default 100
      },

      "level_bounds" : {
        "upper" : float >=0 except for prevalence [0, 1] (required), default 0,

        "lower" : float >=0, <= "upper" (required), default 0
      },

      "increasing" : {
        "age_start" : int[0, 100], <= "age_end" (required), default 0,

        "age_end" : int[0, 100] (required), default 0
      },

      "decreasing" : {
        "age_start" : int[0, 100], <= "age_end" (required), default 0,

        "age_end" : int[0, 100], >= age_start (required), default 0
      },

      "y_maximum" : float > 0 (required), default 1,

      "parameter_age_mesh" : [float, float, ...], numbers are in range[0, 100] increasing (required), default [0,10,20,30,40,50,60,70,80,90,100]
    }
\end{verbatim}

\subsection{Output Template}
\begin{verbatim}
      "data_type" : str (required), one of the following types
                    "i",
                    "p",
                    "r",
                    "f",
                    "rr",
                    "smr",
                    "m_with",
                    "X",
                    "pf", 

      "area" : str (required), a geographic area defined in the area table,

      "sex" : str (required), "male" or "female",

      "age_start" : int[0, 150], <= age_end (required),

      "age_end" : int[0, 150], >= age_start (required),

      "year_start" : int[1990, 2050], current implementation = 1990/2005 or = 1997, <= year_end (required),

      "year_end" : int[1900, 2050], current implementation = 1990/2005 or = 1997, >= year_start (required),

      "age_weights" : [ float, float, ... ] (required*), length equals age_end - age_start + 1,


      additional keys, with corresponding values for all study-level covariates, and all country-level   
      covariates merged for this data_type, this region, this sex, this year_start and this year_end
      starting with x_ if they are included in mean and u_ if the are included in dispersion
      for example:
      "x_time" : float
      "u_biased" : [0,1]
\end{verbatim}

\subsection{Areas Hierarchy}
The hierarchy of areas for the multilevel model, is represented as a
directed graph, which has an underlying undirected graph with no
cycles.  This hierarchy together with the areas to fit determines the
hierarchical structure of the model, i.e. how the model borrows
strength between different geographic areas.

\subsection{Areas to Fit}
The nodes in the hierarchy to generate estimates for.  The estimates
are generated starting from the nodes highest in the hierarchy and
these estimates are used as ``empirical priors'' for nodes lower in
the hierarchy.  In the future, this approach could also determine at
what level consistency in enforced.
