\section{DisMod III Input Data}

The input data files for DisMod III are stored as comma-separated
value (CSV) files and Javascript Object Notation (JSON) files.  This
is conveniently readable by computers, and moderately readable by
humans, and provides a lot of flexibility.  It can prove unwieldly for
very large datasets, but so far it has worked fine.  Support of
compressed JSON files is planned for the day when datasets become too
large to store as plain text files.

Internally the format of the DisMod III input data is an object with $5$ important data attributes:
\begin{verbatim}
  input_data : Pandas.DataFrame of input data (observations and covariates)
  output_template : Pandas.DataFrame of output covariate data
  parameters : dict of model parameter values
  hierarchy : networkx.DiGraph representing the hierarchical similarity assumptions of geographic areas
  nodes_to_fit : list of nodes in the hierarchy to generate consistent estimates for
\end{verbatim}

Each of these entries corresponds to a human-readable file that is
stored locally on disk, with revisions tracked in the git revision
control system.  Each is described in more detail now.

\subsection{Input Data}
The input data comma-separated value (CSV) is a text file of tabular
data specifying all of the data that will go into the likelihood of
the model.

The CSV begins with a row of column headers, and each row of the CSV after the column headers takes the following form:
\begin{verbatim}
      "data_type" : str (required), one of the following types:
                      "csmr" : cause-specific mortality rate = # deaths with icd code matching condition / PY in S+C
                      "f" : excess-mortality rate = m_with - m
                      "i" : incidence rate = # incident cases / PY in S
                      "m" : without-condition mortality rate = # deaths in S / PY in S
                      "m_all" : all-cause mortality rate = # deaths in S+C / PY in S+C
                      "m_with" : with-condition mortality rate = # deaths in C / PY in C = m + f
                      "p" : prevalence = C / (S+C)
                      "pf" : prevalence*excess mortality = p*f
                      "r" : remission rate = # remitted cases / PY in C
                      "rr" : relative risk = (m+f) / m
                      "smr" : standardized mortality rate = (m+f) / m_all
                      "X" : case duration (in years)

      "value" : float (required), parameter value limits
                "csmr"  >= 0,
                "f" >=0,
                "i" >= 0,
                "m" >= 0,
                "m_all" >= 0,
                "m_with" >= 0,
                "p" [0, 1],
                "pf" >= 0
                "r" >= 0,
                "rr" >= 0,
                "smr" >= 0,
                "X" >= 0,

      "area" : str (required), a geographic area defined in the area table,

      "sex" : str (required), one of "male", "female", "total",

      "age_start" : int[0, 150], <= age_end (required),

      "age_end" : int[0, 150], >= age_start (required),

      "year_start" : int[1900, 2050], <= year_end (required),

      "year_end" : int[1900, 2050], >= year_start (required),

      "standard_error" : float > 0 (optional*),

      "effective_sample_size" : int > 0 (optional*),

      "lower_ci" : float >= 0 <= Parameter Value (optional*),
      
      "upper_ci" : float > Parameter Value (optional*),

    (* se is required for X, ci is required for rr, smr, ess is required for all others)

      "age_weights" : [ float, float, ... ] (required), length equals age_end - age_start + 1,
                      default/missing assume to be [ 1, ... ],

      additional keys, with corresponding values for all study-level covariates, and all country-level   
      covariates merged for this data_type, this region, this sex, this year_start and this year_end
      starting with x_ if they are included in mean and u_ if the are included in dispersion
      for example:
      "x_time" : float
      "u_biased" : [0,1]
\end{verbatim}


\subsection{Parameters}
\begin{verbatim}
    param_list = {
      "p" : param_dict (required), see below,
      "i" : param_dict (required), see below,
      "r" : param_dict (required), see below,
      "f" : param_dict (required), see below,
      "X" : param_dict (required), see below,
      "rr" : param_dict (required) see below,
    }

    param_dict = {
      "smoothness" : {
        "amount" : str (required), one of "Slightly", "No Prior", "Moderately", "Very"], default "Slightly",

        "age_start" : int[0, 100], <= "age_end" (required), default 0,

        "age_end" : int[0, 100], >= age_start (required), default 100
      },

      "heterogeneity" : str (required), one of "Slightly", "Moderately", "Very", "Unusable", default "Slightly",

      "level_value" : {
        "value" : float >= level_bounds["lower"], <= level_bounds["upper"] (required), default 0,

        "age_before" : int[0, 100], <= age_after (required), default 0,

        "age_after" : int[0, 100], >= age_before (required), default 100
      },

      "level_bounds" : {
        "upper" : float >=0 except for prevalence [0, 1] (required), default 0,

        "lower" : float >=0, <= "upper" (required), default 0
      },

      "increasing" : {
        "age_start" : int[0, 100], <= "age_end" (required), default 0,

        "age_end" : int[0, 100] (required), default 0
      },

      "decreasing" : {
        "age_start" : int[0, 100], <= "age_end" (required), default 0,

        "age_end" : int[0, 100], >= age_start (required), default 0
      },

      "y_maximum" : float > 0 (required), default 1,

      "parameter_age_mesh" : [float, float, ...], numbers are in range[0, 100] increasing (required), default [0,10,20,30,40,50,60,70,80,90,100]

      "fixed_effects" : {effect_name: effect_dict, ...}
      "random_effects" : {effect_name: effect_dict, ...}
        effect_dict = {dist:str, mu:float, sigma:float, lower:float, upper:float}
        For example:
        {'x_LDI': dict(dist='TruncatedNormal', mu=0, sigma=1, lower=-2, upper=2},
        x_Alcohol,Uniform,,,-.1,.1
        x_Sex,Normal,-.5,.1,,
    }
\end{verbatim}

The value of ``smoothness'' is an influential prior and must be chosen
judiciously.  It is designed to answer the question, ``How different
do I expect the value of this rate to be at age $a+10$ if I know the
value at age $a$?''  For example, if it seems implausible that the
rate at age $a+10$ be more than twice the rate at age $a$, where
``implausible'' is quantified as 1 chance out of 100, then the
smoothness parameter should take value $2/3$, because a normal
distribution with this standard deviation will be less than $2$ with
probability around $.99$.  This is operationalized through a \Matern
covariance function with smoothness parameter chosen to have $C(0,
10) / C(0,0)$ equal to $1$ minus the smoothness value, as
\[
\log \pi(a) - \log \pi(a') \sim \Normal(0, C(a, a')) \quad \text{for } a, a' \in \text{appropriate set TK}
\]

Including more informative priors on fixed and random effects can be
very important for MCMC convergence.  Here is an example of how to
make super-region random effects dispersion prior come from a
truncated normal distribution, with lower bound .01 (meaning more than
3\% variation between super-regions is surprising), and upper bound
.05 (meaning more than 15\% variation between super-regions is
surprising, and median .03, (meaning that a priori I believe that 9\%
variation between super-regions is surprising).

\begin{verbatim}
      "sigma_alpha_i_1": {"mu": 0.03, "sigma": 0.03, "lower": 0.01, "upper": 0.05},
      "BEL": {"dist": "Constant", "mu": 0.0},
      "europe_western": {"dist": "Constant", "mu": 0.0},
      "north_africa_middle_east": {"dist": "Constant", "mu": 0.0}

      "x_hospital": {"dist": "normal", "mu": 0.0, "sigma": 0.0},
      "x_firstever": {"dist": "normal", "mu": 0.0, "sigma": 0.0},
      "x_COD_log_IHD_ASDR_ctry_6_Dec_2011": {"dist": "normal", "mu": 0.0, "sigma": 0.0},
      "x_nottroponinuse": {"dist": "normal", "mu": 0.0, "sigma": 1.0},
      "x_sex": {"dist": "normal", "mu": 0.0, "sigma": 0.0}

\end{verbatim}

\subsection{Output Template}
\begin{verbatim}
      "data_type" : str (required), one of the following types
                    "i",
                    "p",
                    "r",
                    "f",
                    "rr",
                    "smr",
                    "m_with",
                    "X",
                    "pf", 

      "area" : str (required), a geographic area defined in the area table,

      "sex" : str (required), "male" or "female",

      "age_start" : int[0, 150], <= age_end (required),

      "age_end" : int[0, 150], >= age_start (required),

      "year_start" : int[1990, 2050], current implementation = 1990/2005 or = 1997, <= year_end (required),

      "year_end" : int[1900, 2050], current implementation = 1990/2005 or = 1997, >= year_start (required),

      "age_weights" : [ float, float, ... ] (required*), length equals age_end - age_start + 1,


      additional keys, with corresponding values for all study-level covariates, and all country-level   
      covariates merged for this data_type, this region, this sex, this year_start and this year_end
      starting with x_ if they are included in mean and z_ if the are included in dispersion
      for example:
      "x_icd10_diag_crit" : float
      "z_biased" : [0,1]
\end{verbatim}

\subsection{Spatial Hierarchy}
The spatial hierarchy of areas/sexes/years for the multilevel model, is represented as a
directed graph, which has an underlying undirected graph with no
cycles.  This hierarchy, together with the nodes to fit, determines the
hierarchical structure of the model, i.e. how the model borrows
strength between different geographic areas.

Each of the nodes in the hierarchy has attributes which describe the
subpopulation it refers to: a geographic area, sex, year_start, and
year_end, as well as a population estimate for the size of this
subpopulation, which is used in aggregating estimates to higher levels
of the hierarchy.

Each of the edges in the hierarchy has a weight associated with it,
which represents the expert prior on the variation between the nodes
at the beginning and end of the edge.  These are used as a parameter
in a normally distributed prior on the age-by-age similarity
``potential'' for the two regions.  For example, if it seems very
unlikely that an age-specific prevalence for Egypt differs from the
age-specific prevalence of North Africa/Middle East by more than 30\%,
and ``very unlikely'' means you would not accept odds of less than 100
to 1 to make a bet, then the weight on the edge from NA/ME to EGY should be
10 (because standard deviation of .1 yields a 99\% uncertainty
interval of .3).  The exact formulation of the similarity potential is
explored in detail in Section~\ref{TK}, where I justify the standard approach taken
DM3, which is included here for convenience:
\[
\log \pi_{r_1}(a) - \log \pi_{r_2}(a) \sim \Normal(0, \log(1 + w(r_1, r_2))^2)\quad \text{for } a = 0, 10, 20, \ldots, 100
\]

\subsection{Nodes to Fit}
The nodes in the hierarchy to generate estimates for.  The estimates
are generated starting from the nodes highest in the hierarchy and
these estimates are used as ``empirical priors'' for nodes lower in
the hierarchy.  In the future, this approach could also determine at
what level consistency in enforced.
