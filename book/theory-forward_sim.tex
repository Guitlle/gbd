h1. Forward Simulation Examples

When a generic disease model is initialized with all-cause mortality data and nothing else, the initial values produce the following set of consistent age patterns:

!initial.png!

The next example shows that increasing the remission rate with incidence and excess mortality unchanged (and all-cause mortality unchanged as well) leads to a very different age pattern for prevalence. It is also worth pointing out that since prevalence has changed with excess mortality and all-cause mortality rates held constant, the with-condition and background mortality rates have also changed to maintain consistency.

!more-remission.png!

By changing the incidence rate age pattern to be increasing as a function of the square root age, I can demonstrate that very similar prevalence rates are consistent with very different incidence and remission rates.

!increasing-incidence.png!

Although the prevalence age pattern is largely determined by the remission, incidence, and mortality rates, the birth prevalence can also change the shape dramatically.  Here are the results of the same remission, incidence, and mortality rates as above, but with a birth prevalence of 20%.

!birth-prevalence.png!

An interesting, and perhaps unexpected feature of this set of consistent rates above is that when prevalence levels start so high, the levels remain high during the teenage years, where all-cause mortality rates are quite low in this population (following the rates for males in the North American High Income region in 2005), which produces very low levels of background mortality for these age groups. In other words, in order for the model to be consistent, it is necessary to assume that the vast preponderance of teenage deaths are due to this disease.

Finally, the excess mortality rate (which is the most difficult of the rates to conceptualize, due to its unobservability) has an effect of modulating the prevalence that is similar to the remission rate, although not identical.  The final figure in this series shows the results of choosing an excess mortality rate to have an age function equal to ten times the all-cause mortality rate (which is to say a standardized mortality rate of constant value eleven for all ages).

!higher-smr.png!

The decreasing prevalence after age 65 is worthy of remarking on. Although the incidence rate is increasing and the remission rate remains unchanged, having a constant (albeit high) standardized mortality ratio means that when all-cause mortality rises, the with-condition mortality rises differentially with such magnitude that the prevalence of the condition in older populations goes down.

To summarize, this series of figures has shown the intuitive and less-than-intuitive way that the levels and age patterns of different epidemiological parameters must be interrelated to satisfy the fundamental equations of population health (when disease rates for each age are changing negligibly slowly as a function of time).

The next series of figures continues to build familiarity with the features of consistent disease modeling, by selecting age patterns for certain rates based on toy models of a variety of diseases.  For example, for a disorder like depression, for which there is primarily incidence in early adulthood, high remission rate, and low excess mortality, the consistency conditions produce a prevalence age pettern that peaks at age 25:
!forward-sim-mental.png!

For a congenital disorder, like TK, with birth prevalence, no incidence after birth, no remission, and substantial mortality, the consistent prevalence age pattern is the following:
!forward-sim-congenital.png!

For a disorder that affects the elderly, like TK, the consistent age patterns for mortality, incidence, remission, and prevalence could look like the following:
!forward-sim-old_age.png!

And for a disorder related to the reproductive system, like TK, with substantial excess mortality and incidence during ages 15-60, and remission that increases substantially at age 55, the consistent age patterns could look like the following:
!forward-sim-reproductive.png!

To conclude this series of plots, I've included an "incidence impulse response" example, showing the prevalence produced to be consistent with an incidence pattern that is only nonzero for a single age group:
!forward-sim-incidence_pulse.png!

This also provides a mechanism to investigate how wrong the estimates may become when the assumption that rates are constant over time (for a given age) is violated. This is the core of my simulation approach to model validation, to which I will return in section TK.

TK The simulation study approach can be described in full detail here as well, and can serve as justification for decisions described in the next two chapters.

