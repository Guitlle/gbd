\section{History of generic disease modeling}

This line of research into meta-analytic techniques for descriptive
epidemiology has been accompanied by software implementations since
the 1990s.  The development and refinement of these computer codes
provides conveniently named milestones through the history of the
approach.  For example, DisMod I was software developed in the early
1990s to support analysis in the original Global Burder of Disease
Study.  Computing power has increased dramatically over the 20-year
period in which the DisMod family of generic disease modeling software
has evolved, and the aspiration of methods has expanded as well.

The precursor to the first DisMod, the Harvard Incidence-Prevalence
(HIP) Model, was a spreadsheet implemented in Lotus 123.
\cite{Murray_Quantifying_1994} This model took as input a set of
instantaneous incidence, remission, and excess mortality rates for $5$
age groups, and produced estimates of disease prevalence and duration.
The model involved constructing a life table to simulate a cohort
exposed to a set of age-specific incidence, remission, case fatality
and background mortality hazards. At each year in the life table, the
model simulated a simple 3-compartment model to provide estimates of
the number susceptible, the number of cases and the number of deaths
to input into the life table for the next year.  It was used primarily
for three purposes: to find prevalence for conditions where incidence
is known and reasonable assumptions about remission and excess
mortality can be made, to find attributable deaths that were not
directly coded to a specific cause, and to find incidence for
conditions where prevalence is known.  The third use case required an
interative procedure in the HIP model, since the input incidence was
unknown.

As is often the case in science, a very similar approach had been
developed previously by researchers, at the International Institute
for Applied Systems Analysis in Austria in the 1970s. This work was
part of a broad program to develop a generic Healthcare System Model
to improve management and planning in the health sector. One component
of this model was a computer program to estimate prevalence from
incidence \cite{Klementiev_On_1977}. That program evolved a population
exposed to age-specific incidences of disease and death through time.
Although it was designed specifically for terminal illness, it is
similar to the DisMod line of models in many ways. It was applied to
estimate the prevalence of malignant neoplasm in Austria, France and
Belgium.


Over the course of the first Global Burden of Disease Study, the HIP
Model evolved in DisMod I.\cite{Harvard_Global_1996} This was
formalized as a four compartment model, and corresponding system of
differential equations.  Like the HIP model, the input to DisMod I
consisted of instantaneous rates for incidence, remission, and excess
mortality, now specified for $9$ age groups.  In addition to the three
use cases from the HIP model, DisMod I was also used to estimate the
average duration of disabling sequelae, as a function of age.

DisMod II moved from forward simulation into the realm of
optimization.  It provided more control over inputs, as well as a a
graphical user interface and comprehensive user manual, making it more
widely useable than previous iterations.\cite{Barendregt_Generic_2003}
In addition to accepting input of instantaneous rates for incidence,
remission, and excess mortality, DisMod II was also capable of using
age-specific prevalence and cause-specific mortality rates, as well as
incidence as a population rate, and duration when it is short (less
than one year).  It also provided an algorithmic method for data
confrontation, wherein the downhill simplex method was used to
minimize the weighted difference between the inputs and the output
predictions.

The generic disease modeling approach was distributed without cost by
the World Health Organization (WHO) through the freely available
DisMod II software. It has been used widely in burden of disease
studies over the last 15 years. These studies adopted the methodology
of the global study, but aimed to assess burden at a level of detail
more relevant for national policymakers. Over two dozen countries have
undertaken a national burden of disease study.\ref{TK WHO 2003?} The
first generation of these studies, which used the same valuations for
the disability weight of living in different health states as the
original Global Burden of Disease study, included studies in Mexico,
Chile, Colombia and
Mauritius.\cite{Lozano_Burden_1995,Concha_Carga_1996,Republica_Carga_1994,Vos_Mauritius_1996}

Despite its wide application, DisMod II has not been without
criticisms.  One methodological concern that emerged from extensive
application of the model centered on the difficultly in producing
consistent estimates that exhibited face validity, for example age
patterns that increased monotonically as a function of age. Despite
strongly held prior beliefs on the part of domain experts, it was not
uncommon for the data to show oscillations as a function of age, due
to the contortions that DisMod 2 would subject rates to in order to
produce consistent estimates as close to the inconsistent input
estimates as possible.

Another important challenge in the DisMod II workflow was the
production of single best estimates for at least three independent
rates to be used as input.  Systematic review often finds multiple
measurements of an age-specific rate, and only one can be the input to
DisMod II.  Transforming a large collection of measured values, often
for incommensurate age intervals, to a single best estimate of disease
prevalence was a difficult challenge in analysis that is a necessary
preprocessing step to do meta-analysis with DisMod II.

Finally, although DisMod II excelled in providing consistent estimates
from the confrontation of inconsistent estimates of several disease
parameters for a single place and time, it was laborious on the part
of the data analyst to produce comparable estimates for a variety of
different places and times. In the Global Burden of Disease (GBD) 2010
Study, there are 21 geographic regions to produce estimates for, at
three different points in time, for males and females. Even an
analysis that is trivial for one region/time/sex becomes burdensome
when it must be replicated $126$ times.

DisMod III, a complete redevelopment of the method initiated for the
GBD 2010 Study, continued the trend towards including more formal
inferential techniques into the estimation process.  The broad
principle behind this next generation of generic disease modeling can
be characterized as connecting a system dynamics model to a
statistical model, so that instead of doing forward simulation, as is
traditionally the case in system dynamics modeling, the model
describes an inverse problem. This approach is termed Integrative
Systems Modeling (ISM) and it is emerging as a powerful approach for
developing models that integrate all available datasources.  ISM, and
its intellectual history, is the topic of the the next section of this
chapter.  On top of the compartmental model initially conceived for
the HIP model, DisMod III layers a negative binomial mixed effects
spline model, which is fit directly to the data extracted in
systematic review using Bayesian methods.  The details of this
approach constitute the bulk of the first half of this book.

\section{Integrative systems modeling}
\label{intro-ism}
A vast body of literature exists on compartmental modeling and its
wide applicability to modeling the dynamics of complex systems
\cite{Forrester_Principles_1968, Meadows_Thinking_2008,
  Bossel_Systems_2007}.  TK more words in the way of a general
introduction to this idea.  Something about Forrester's outsider
models of epi, something about the introductory textbook in
environmental science Consider a spherical cow, etc.

\subsection{Compartmental models in epidemiology}
In epidemiology, compartmental models are often constructed to
simulate infectious disease dynamics.\cite{Anderson_Infectious_1991}
The classic Susceptible-Infectious-Recovered (SIR) model evolves a
population through a Susceptible compartment to an Infected
compartment to a Recovered
compartment.\cite{Kermack_Contribution_1927} Infection dynamics are
captured by making the amount of mass that moves from the Susceptible
compartment to the Infected compartment dependent on the product of
the masses in the two compartments. This dependence implies that the
number of new infections will increase with the number of current
infections. Extensions of this basic model abound.\cite{Daley_Epidemic_2001,Brauer_Mathematical_2001} The transition
parameters in this class of compartmental models, incidence and
remission for instance in the case of the SIR model, are usually set
based on extracting point estimates of the parameters from literature
reviews. Uncertainty is usually assessed based on a sensitivity
analysis that solves the compartmental model for the range of
parameter estimates found in the literature (UCLA disease modeler who
started latin hypercube sampling) \cite{Nagelkerke_Modelling_2002,
  Brandeau_Screening_1993, Broutin_Impact_2010}.

TK less harsh version of the following paragraph: In the vast majority
of statistical analyses, this estimation approach would not be
considered sufficient. Instead, the analyst would attempt to find the
parameters that, for instance, maximized the likelihood of a set of
data samples of the parameter values. This statistical approach has
the advantage that uncertainty can be rigorously quantified and an
optimal estimate can be identified based on a transparent model.

Combining these approaches is currently the subject of basic research.
TK references to the ``plug-and-play'' approach to statistical
inference for mechanistic models.

Advances in statistical modeling and computation have allowed
increasingly sophisticated models to be fit to data. These advances
have spawned a new modeling approach that seeks to provide more
reliable point estimates and estimates of uncertainty for parameters
in compartmental models. This new approach, integrative systems
modeling, connects a system dynamics model to a statistical model so
that parameters in the system can be estimated in a statistical
framework without sacrificing the structure provided by the dynamical
model.

The analyst building a statistical model has a rich vocabulary with
which to describe the data-generating process of interest. Data can
come from a range of distributions. Hierarchical data can be expressed
via random effects and smooth data through the correlation structure
of a covariance matrix. In the most mature forms of integrated systems
modeling, this rich vocabulary is made available for estimating
parameters in a compartmental model. DisMod III is a prime example of
connecting a sophisticated statistical model to the generic disease
dynamics model. The complexity of the statistical model and the
complexity of the underlying dynamic systems model vary across
different applications.

\subsection{Compartmental models in pharmacokinetics}
The field of pharmacokinetics and pharmacodynamics (PK/PD) provides
some of the most sophisticated examples of connecting a statistical
model to a compartmental model.

Pharmacokinetics is the study of how drugs get absorbed and
distributed in the body. Pharmacodynamics is the study of the effect
of drugs on the body. Much of the content of these two fields overlap
so they are often studied together. Within PK/PD, the field of
population pharmacokinetics attempts to understand the sources of
variability in drug response among individuals
\cite{Yuh_Population_1994}. Because clinical trials provide data on
only a small subset of the target patient population and at small
sample sizes, it is often difficult to estimate variation among
individuals without imposing additional structure on the estimation
problem. In 1972, the field of population pharmacokinetics began in
earnest when nonlinear mixed effects modeling was proposed as a
solution to the limited clinical data
\cite{Sheiner_Modelling_1972}. The techniques that have emerged in
this field are mathematically identical to the new generation of
generic disease modeling illustrated by DisMod III. Analysts in
population pharmacokinetics connect a random effects model of patients
within a study (the statistical model) to a compartmental model that
describes the process of a drug's absorption in the body (the dynamic
systems model).

Analogous to the advent of the DisMod software for simulating and
estimating generic disease models, many different software packages
have arisen to help conduct analyses in population
pharmacokinetics. NONMEM, which developed at UCSF, was one of the
first.\cite{Beal_NONMEM_2009} SAAM II, a computer tool for the
simulation, analysis and modeling of pharmacokinetic data, also allows
users to fit compartmental models of the drug response to clinical
data using the integrative systems modeling
approach.\cite{Barrett_SAAM_1998}

TK a little more about this, and its glories.




