\section{History of Generic Disease Modeling}

DisMod, the original generic disease modeling system was developed
theoretically by Chris Murray (?) and implemented by TK in TK as part
of the original Global Burden of Disease Study in TK. TK Some
highlights of the original DisMod. Little was recorded about this
approach, although it is remembered fondly to this day by the
researchers involved.

For the second iteration of the Global Burden of Disease Study, Jan
van der TK, developed a new implementation of DisMod in TK Pascal,
which provided a stable user-interface and sufficient documentation
for the application to be used widely beyond the initial analysis.
DisMod II, like the original DisMod focused on data confrontation, and
specialized in combining single best-estimates of individual disease
parameters, such as incidence, prevalence, remission, and
case-fatality to produce a set of internally consistent estimates for
a single time, place, and sex.

This approach was distributed publicly by WHO and used widely in
burden of disease studies. TK some highlights, mentioning Theo Vos by
name. However, there were some feature requests and methodological
concerns that developed over the decade that DisMod II reigned as the
standard approach for disease burden estimation.

Chief among them was the difficultly in producing consistent estimates
that exhibited ``face validity'', for example age patterns that
increased monotonically as a function of age.  Despite strongly held
prior beliefs on the part of domain experts, it was not uncommon for
the data to show oscillations as a function of age, due to the
contortions that DisMod II would subject rates to in order to produce
consistent estimates as close to the inconsistent input estimates as
possible.

Another important challenge in the DisMod II workflow was the
production of single best estimates for at least 3 independent rates
in the first place.  As mentioned in section TK, for common diseases
like TK, there are over TK studies of disease prevalence that met the
inclusion criteria of a recent systematic literature review. DisMod II
provided no guidance on how to go from this large collection of
estimates, often for incommensurate age intervals, to a single best
estimate of disease prevalence.

Finally, although DisMod II excelled in providing consistent estimates
from the confrontation of inconsistent estimates of several disease
parameters for a single place and time, it was laborious on the part
of the data analyst to produce comparable estimates for a variety of
different places and times. In the Global Burden of Disease 2010
Study, there are 21 geographic regions to produce estimates for, at 3
different points in time, for males and females. Even an analysis that
is trivial for one region/time/sex becomes burdensome when it must be
replicated 126 times.

It was for these reasons that in 2008 I began a journey to develop a
new generation of generic disease modeling system. This work was
inspired by the previous generations of generic disease models
developed in the past global and national burden of disease studies,
as well as work on mortality estimation and prediction by Girotis
(SPELLING TK) and King, which used Bayesian methods to estimate age
patterns of mortality simultaneously for multiple regions of the
world. The model was developed in collaboration with many disease
experts who volunteered to help in the GBD 2010 study, an effort that
quickly because quite a massive undertaking.

The mental disorders group and substance dependence group were
particularly punctual in producing preliminary results from their
comprehensive reviews of the published and unpublished literature, and
therefore became the test examples in my development process (along
with a synthetic test example I developed, which I came to
affectionately call ``chronic dismoditis'').

Over the course of the 3 years I spent developing this approach, it
became clear that there is a broader principle which is behind the
model.  This can be characterized as connecting a system dynamics
model to a statistical model, so that instead of doing forward
simulation, as is traditionally the case in system dynamics modeling,
the model describes an inverse problem. This sort of inference on
compartmental models is rare in epidemiology, despite the long-running
focus of infectious disease modeling on compartmental formulations of
disease dynamics, and recently increased emphasis on ``model
calibration''. And to the best of my knowledge, it is rare in other
fields as well. My extensive web-searching discovered a recently
developed line of research into Bayesian calibration of computer
models, a handful of papers on ``plug-and-play'' modeling, and recurring
pleas that someone actually start doing inference on compartmental
models. I should also mention that there is an application to disase
modeling which I was eventually introduced to which predates the
global burden of disease motivated work by TK years, TK DESCRIPTION OF
THE RUSSIAN CANCER MODEL.

However there is only one field that I came across which has
systematically embraced the approach that I have taken here, which is
clinical pharmacokinetics. Since the 19XXs TK, PK/PD analysis has
developed appropriately complex compartmental models of human and
animal physiological systems and used measurements of pharmacological
data to fit the model parameters. The techniques emerging from this
field for so-called Population PK are mathematically identical to the
new generation of generic disease modeling and extensions that I have
stumbled upon. Connecting with the applied mathematicians like Brad
Bell, who worked on the initial development of the System for
Population Kinetics, have greatly assisted my understanding of the
technical issues at work in the generic disease model, and promise
exciting extensions in the future.

This book focuses on the statistical model and computational method
behind my new generic disease model, together with a series of
extensive case-studies of the model in action, culled from the massive
labors of the GBD2010. However, the computational infrastructure that
was required for the task should not be overlooked. Modern Bayesian
statistical methods are computationally intensive, and sharing
preliminary results with a globally distributed team of over 800
experts presents unique communications challenges, as well. Without
the existence of the easy-to-use web application development framework
Django, the statistical models and computational methods would have
been useless because no one would have been able to make use of them
in the timeframe available.  Similarly, without the expert assistance
of system adminstrators Trey TK and Serkan Yalchin, as well as
software engineer Jiaji Du and web designed (cum epidemiologist) Ben
Althouse, the models and methods would never have seen the light of
day.

Finally, the Bayesian models and methods used in this project relied
exclusively on the Python PyMC package [TK REF], developed by Chris
Fonnesbeck, David Huard, and Anand Patil. It would have been simply
impossible for me to put this together without their free/libre
open-source software package.
