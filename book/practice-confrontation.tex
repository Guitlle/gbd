\section{Data confrontation}
This section explores several diseases where systematic review
resulted in enough data from different sources to see data
confrontation in action.  This might be because there are studies on
incidence and prevalence, or because the incidence is all at birth,
and there is no remission.  Whatever the setting, when data from
different sources are integrated together through the system dynamics
model, this is interesting.

\subsection{Epilipsy}
16158
16261

\subsection{MS}
16173

\subsection{Dementia}
16172

\subsection{Parkinsons}
16766

\subsection{AF}
15681

\subsection{Arthritis}
15993
15988

With the assumption that there is no remission, this model is
constrained enough to permit data confrontation.

Without mortality data, such a model would appear not to be over
constrained.  However, even with no conditions on mortality, the model
has an implicit assumption that arthrisit has no protective effects
(an assumption that experts are not upset by).  In this case, it does
lead to confrontation, because the incididence data is not large
enough to match the prevalence data, even if excess-mortality is zero.

TK discussion of how the assumption that disease is not changing over
time affects this, and how we can use google trends to obtain evidence
for or against this assumption.


\subsection{Depression in North America}
12539 
16152


\subsection{Anorexia}
16265
