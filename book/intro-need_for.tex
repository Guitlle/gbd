\section[Need for Confrontation]{The Need For Data Integration and Confrontation in Generic Disease Modeling}

A review of the published and grey literature that exists for any
disease of interest will reveal a variety of information, the results
of studies conducted for many different reasons, by many different
people, at many different times. When developing estimates of disease
burden, and particularly when trying to estimate YLDs for a burden of
disease study, I do not want to overlook any of the results of this
myriad of studies without a good reason. However, the techniques of
standard meta-analysis are very far from sufficient. A systematic
literature review for DISEASE EXAMPLE TK found TK studies of
prevalence from TK COUNTRY, conducted during TK TIME PERIOD.  However,
the age ranges of these studies varied so significantly that only TK
NUMBER applied to the same subpopulation.  The situation is even more
complicated when considering data in a global setting. It is a
reasonable hypothesis that the age patterns of DISEASE TK are quite
similar for COUNTRY TK1 and COUNTRY TK2, and less similar but still
not completely different in COUNTRY TK3. This book will develop from
first principles a meta-analytical approach to integrate data from
different studies, collected from different geographical regions,
different and overlapping age groups, all at different times, to
produce estimates of age-, time-, sex-, and region-specific
epidemiological disease parameters such as incidence, prevalence,
remission, and relative risk of mortality.

A comprehensive review of published and grey literature for many
diseases will also reveal that multiple, related disease parameters
have been studied and can be analyzed. For DISEASE TK, PERHAPS
DEMENTIA, studies measuring disease prevalence are common, TK studies
met the inclusion criteria in a recent systematic review. But studies of
disease incidence have also been conducted, as well as studies of
relative risk of mortality and cause-specific mortality. All of these
estimates are related by a logical requirement of internal
consistency.  A prevalent case of the disease can only exist if there
was an incident event sometime in the past, and the number of
prevalent cases this year can be determined from the number of
prevalent cases last year, after adding in all of the incident cases
and subtracting out all of the deaths and remissions (if there is
remission from the disease under consideration).  This suggests a
fundamental equation of population health, which can be further
refined to take age as well as time into account in a systems dynamics
model with 4 compartments shown below.

TK FIGURE OF SYS DYMANICS MODEL

The precise mathematical representation of this model will be
elaborated on in great detail in chapter TK. Here, I will only comment
that intuitively, there must be a relationship between incidence,
prevalence, remission, and with-condition mortality, and the data that
has been gathered for each of these epidemiological parameters should
all be brought together in a theoretically grounded process of data
confrontation to produce a best estimate and ADJECTIVE TK uncertainty
of disease incidence and prevalence, if this is to be used in YLD
estimation. Through this grand synthesis of a meta-analysis, I hope to
produce the best possible estimates of disease burden.

TK HONEST EXAMPLE OF WHAT WE COULD POSSIBLY DO WITHOUT THIS APPROACH
WITH THE DATA AVAILABLE, E.G. TAKE THE MEDIAN OF THE RESULTS.

