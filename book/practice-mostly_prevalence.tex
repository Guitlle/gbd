\section{Modeling diseases with mostly prevalence data}

A surprising headline of the first GBD study was the unexpectedly
large burden of mental disorders like depression and
schizophrenia. The epidemiology of these conditions, as well as many
other mental and neurological disorders, are known primarily from
prevalence studies. This chapter considers in detail the setting where
prevalence data is the primary source of input data.

Several gynecological disorders had systematic reviews which resulted
in only a small amount of prevalence data, and no other measurements
of epidemiological rates like incidence, remission, or mortality.  (TK
really no studies on remission?)

TK A lot of prevalence, expert priors on everything else

TK Prevalence plus a few data points on other things

TK Prevalence data which clearly violates time equilibrium assumption

TK Birth prevalence data only -- no need for fancy statistical model

\subsection{PMS}
\begin{figure}
\begin{center}
\includegraphics[width=\textwidth]{pms-prev.pdf}
\end{center}
\caption{(a) All prevalence data collected in systematic review; (b)
  Prevalance data for North America, High Income region only.  Note
  the extremely high heterogeneity of the data, ranging from
  $<<d['pms.json|dexy']['min_rate_per_100']>>$ per $100$ to
  $<<d['pms.json|dexy']['max_rate_per_100']>>$ per $100$.  Covariate
  modeling can help explain this variation.}
\label{pms-prev}
\end{figure}

Figure~\ref{pms-prev}a shows all of the data collected in systematic
review of the descriptive epidemiology of premenstural syndrome.
Since the review found only data on disease prevalence, and even that
is sparse (available for only $<<d['pms.json|dexy']['num_regions']>>$
of $21$ GBD regions) and extremely noisy, choices made in the modeling
process will have a significant effect on the resulting estimates.
This highlights the importance of making it clear what the model assumptions
are, and how sensitive the results are to these assumptions.

The data gathered in systematic review of premenstural syndrome show
just how sever the situation can be: for any possible prevalence
level, you could find a study where the measured prevalence is within
5 per 100 that satisfied the inclusion criteria for systematic review.
Even restricting our attention to data from the North America, High
Income GBD region (USA and Canada) relevant to the 2005 time period
still yields a data set with no clear age pattern (horizontal bars in
Figure~\ref{pms-prev}b).

The prudent solution in this situation would be to give up, and not
even produce estimates of PMS prevalence, let alone estimates of incidence or
duration.  For a well-funded researcher who really must know the
answer, proper recourse would be to field a study designed to measure
exactly the quantity of interest in the population of
interest.  Since all of the wildly varying data collected in
systematic review shows that the prevalence is at least
$<<d['pms.json|dexy']['min_rate_per_100']>>$ per $100$, this study
need not be enormous.

However, we cannot opt out of estimating and we cannot wait for a new
definitive study to be conducted (although we can use the results of
the systematic review to decide which studies are highest priority in
the future).  This is why model-based estimates are our only option.
We must take caution, though, because, as already mentioned, the
assumptions behind the model will have a dramatic influence on the
estimates.

The assumptions on age patterns that experts in these diseases have
agreed to are the following: there no incidence or prevalence of
disease before age 15 or after age 50, there is no excess mortality,
and there is no remission before age 40.  The specific years in these
assumptions were chosen for parsimony, as they correspond breaks and
midpoints of the age groups to be estimated in the GBD 2010 study,
which include groups for 10-15, 15-20, 20-25, 25-35, 35-45, and 45-55.

With these restrictions to the age patterns in place, the consistent
fit for the data for the NAHI region, 2005 time perior is shown in
Figure~\ref{pms-consistent}.
\begin{figure}
\begin{center}
\includegraphics[width=\textwidth]{pms-consistent.pdf}
\end{center}
\caption{Consistent fit of all data collected in systematic
  review relevant to NAHI 2005, together with expert priors on age pattern.  The expert
  priors are: no incidence or prevalence of disease before age $15$ or
  after age $50$, there is no excess mortality, and there is no
  remission before age $40$.  Knots in the piecewise constant Gaussian
  process model evenly spaced at $5$ year intervals between the ages $15$ and $50$.}
\label{pms-consistent}
\end{figure}

This model is sensitive to the coarse knot selection in the age
pattern model, which must be chosen finely enough to allow consistent
fits that respect the expert priors on when age-specific rates are
non-zero.  Choosing an age pattern model with knots evenly spaced in
5-year intervals is compatible with the model assumptions that there
are changes between zero and non-zero rates at ages 15, 40, and 50.
Knots forming a coarser mesh speed up computation time, but must be
choosen to be compatible with the zero-rate assumptions.  For example,
taking knots at $\{0,15,40,50,100\}$ is the coarsest mesh that could
possibly respect the zero-rate assumptions.  However, the model
assumption that the prevalence is linear between age 15 and 40 results
in a model that cannot represent the rapid changes in human physiology
during puberty.  On the other hand, taking knots on a finer mesh gives
the model more flexibility in representing the age pattern (which
might not be a good thing, considering how noisy the data appears).
Figure~\ref{pms_grids} compares the effects of fitting the data from
North America, High Income to finer and coarses grids.
\begin{figure}
\begin{center}
\includegraphics[width=\textwidth]{pms_grids.pdf}
\end{center}
\caption{Knots in the age pattern models with different patterns. (a)
  coarsest grid with knots at $\{0,15,40,50,100\}$; (b) coarse grid
  with knots at $\{0,15,20,40,50,100\}$; (c) fine grid with knots at
  $5$-year intervals between $15$ and $50$ (as well as at $0$ and
  $100$); (d) fine grid with knots at $2$-year intervals between $15$
  and $50$ (as well as at $0$ and $100$).}
\label{pms_grids}
\end{figure}

Although choosing between these knots in the model can be based on
expert judgement, there are objective measures of model fit, discussed
in Section~\ref{theory-model_checking}, such as in-sample or
out-of-sample predictive accuracy and graphical checks of the
posterior predictive distribution. TK comparison of DIC values for the
different knots, such as $<<d['pms_grids.json|dexy']['results']['dic_a']>>$.
The effects of different priors on the over-dispersion parameter are shown through posterior predictive checks in
Figure~\ref{pms_ppc}.  Here we have include TK potentially relevant study-level
covariates as predictors for the level of the prevalence rate.  TK
discussion of lack of appropriate country-level covariates, drawing on
discussion on Mayo clinic or Lancet articles, lack of significance of study-level covariates for NAHI. The posterior distribution for the over-dispersion parameter in
these models has a value of
$<<'%.1f'%d['pms_ppc.json|dexy']['results']['posterior_dispersion_0']['quantiles']['50']>>$
without covariates and
$<<'%.1f'%d['pms_ppc.json|dexy']['results']['posterior_dispersion_1']['quantiles']['50']>>$
with covariates.

\begin{figure}
\begin{center}
\includegraphics[width=\textwidth]{pms_ppc.pdf}
\end{center}
\caption{Posterior predictive checks for NAHI with and without
  study-level covariates. TK details on the particular study-level covariates, and the posterior distribution of the effect coefficients.}
\label{pms_ppc}
\end{figure}

This provides a reasonable starting point, but we might hope that a
model with more information about systematic differences between the
studies can make more precise predictions.  In particular, there are
many variations in the study methods in the studies gathered by the
systematic review.  Different studies used different case definitions,
studied different subpopulations, and varied in other ways that were
noted when conducting the data extraction.  Including this information
in the model when predicting the level or the dispersion of data as
covariates, as discussed in Section~\ref{TK}, does not change the
predictions for NAHI 2005, however.  When fitting data from other GBD
regions, the effect coefficient for ICD10 case definition vs any other
has a ``significant'' effect, with ICD10 case definitions leading to
measurements of TK\% higher, on average.

TK discussion of producing estimates that differ by region and  by time
using an empirical bayes approach, with the world estimate as the
empirical prior.

TK discussion of the possibility of an acceptible result, although it
is important to investigate the effects of smoothing priors,
heterogeneity priors, level bounds.

TK discussion of when to start smoothing priors, e.g. a fine age mesh
together with a smoothing prior that starts after menarch, leads to
very different age pattern, and possibly different levels overall.


\subsection{Cannabis Dependence, also use}
14390 15303 

16153 16160 

\subsection{Bipolar}
The systematic review for epidemiological rates related to bipolar
disorder came up with TK rows of prevalence data, and TK rows of
standardized mortality ratio (SMR) data.  Since there is so much more
prevalence data than SMR data, I have pooled all of the rows of SMR
data across studies and applied them assuming that SMR for bipolar
does not vary by region or time.  There may be sufficient data to assume
that it \emph{does} vary by sex, however. 

The approach I followed in this setting is one that has come up quite
frequently.  Generate an empirical prior on prevalence, by pooling all
of the world's data in a model with effects for region, sex, and
time, and then generate region/sex/time specific posterior estimates
by applying the consistent model for the appropriate subset of the
prevalence data, together with all the rows of SMR data, and with a
minimal, defensible set of expert assumptions to fill in the remaining
model flexibility, in the case, the assumption that the remission rate
is at most .05 per person-year.

There are two empirical bayes approaches that I described in detail in
Chapter~\ref{TK}, and this model provides a good place to contrast the
results.  Fitting only prevalence data as an empirical prior is
quicker, and provides predictions of prevalence age patters for
regoins with only a few rows of data for large age groups (as well as
for regions with no rows of data).  Fitting the whole world's data
consistently for the model data provides a slightly different result,
because it incorporates the relationship between incidence,
prevalence, remission, and excess mortality from the start.

TK what mortality pattern is used by the consistent world model?

\subsection{Depression outside of North America}
12539 
16152
 The prevalence data, on the
other hand, does cover the majority of the 21 regions that partition
the world in the GBD2010 study, albeit quite non-uniformly.  There are
over 100 rows of data about each of North American High Income and
Western Europe, while there are less than 10 rows of data about Latin
America, Center; Latin America, Southern; and Sub-Saharan Africa,
Southern; and no rows of data at all about 4 other regions.

\subsection{Hepatitis C}
Diseases with strong cohort effects, like hepatitis C in North America,
violate the assumptions from Section~\ref{TK} enough to
necessitate a special approach.  That is what I investigate in this
section.

Figure~\ref{hep_c-data}a shows the problem: there is a peak in
prevalence for the 30-50 year olds, yet there is a strong belief among
experts that the decline in prevalence seen in 50-60 year olds is not
due to remission or excess mortality.  Experts know that there is
simply not enough remission or excess mortality to explain this
decrease because TK.  Instead, disease experts believe that this age
pattern is the result of a ``cohort effect'', where a spike in
incidence was caused by factors like unsafe injections and hightened
intraveneous drug use during the 1970s.  This spike differentially
affected an at-risk age group, which is now the cohort in the cohort
effect.  This group with hightened prevalence is now aging through the
population, as the 15-25 year olds of the 1970s become the 35-45 year
olds of the 1990s.  This phenomonen violates the ``time stationarity''
assumption from
Section~\ref{theory-forward_sim-compartmental_model-simplying_assumptions},
and does so to such a degree that any incidence rates generated from
this prevalence data would be completely unbelievable.  Even fitting
the prevalence data to be consistent with expert derived bounds on
remission and mortality levels is not possible, as demonstrated in
Figure~\ref{hep_c-consistent}b.
\begin{figure}
\begin{center}
\includegraphics[width=\textwidth]{hep_c-smoothing.pdf}
\end{center}
\caption{Age-pattern of Hepatitis C prevalence in USA 1990.  Panel (a)
  shows the relevant data collected from systematic review. The width
  of each horizontal bar shows the age group and the vertical line
  through the middle of each bar shows the $95\%$ uncertainty
  interval.  Panels (b), (c), and (d) show estimates of the prevalence
  age pattern for three different levels of smoothing.  Panel (b)
  shows the results of the expert belief that the age pattern is
  ``slightly smooth'', which sets
  $\rho=<<d['age_pattern_covariance.json|dexy']['rho']['slightly']>>$.  Panel (c)
  shows the ``moderately smooth'' expert prior, which sets
  $\rho=<<d['age_pattern_covariance.json|dexy']['rho']['moderately']>>$.  Panel (d)
  shows the shows the ``very smooth'' expert prior, which sets
  $\rho=<<d['age_pattern_covariance.json|dexy']['rho']['very']>>$.}
\label{hep_c-data}
\end{figure}

When the system dynamics model is removed and only the age integrating
negative binomial rate model for prevalence remains, for rows of data
with the form $(p_i, n_i, r_i, s_i, t_i, a_{0,i}, a_{1,i}, X_i,
\boldw_i)$, where $p_i$ is the measured prevalence value, $n_i$ is the
effective sample size, $r_i, s_i,$ and $t_i$ are the region covariate
(categorical), sex covariate (ordinal), and time covariates
(continuous, normalized), $(a_{0,i}, a_{i,1})$ is the age group, $X_i$
is the ``bias'' covariate, and $\boldw_i$ is the age weight function,
the full specification of the model is the following
\begin{align*}
p_i n_i &\sim \NegativeBinomial(\pi_i n_i, \delta_i)\\
\pi_i &= \int_{a_{0,i}}^{a_{1,i}} \boldpi_{r_i,s_i,t_i,X_i}(a)d\boldw_i(a)\\
\boldpi_{r,s,t,X}(a) &\sim \PCGP(\{a_1, \ldots, a_A\}; \boldmu_{r,s,t,X}, \scC_\rho)\\
\boldmu_{r,s,t,X}(a) &= \exp\left\{\alpha_r + \alpha_{sex} s + \alpha_{year} t + 
\beta X + \gamma_{r,s,t,a}\right\} \qquad \text{for }a \in \{a_1, \ldots, a_A\}\\
\calC_\rho &= \Matern(\nu=2, \sigma=10, \rho)\\
\alpha_{sex} &\sim \Normal(0, TK)\\
\alpha_{year} &\sim \Normal(0, TK)\\
\alpha_{r_1}, \alpha_{r_2}, \ldots, \alpha_{r_R} &\sim \Normal(0, \Sigma_{regions})\\
\beta &\sim \Normal(0, TK)\\
\gamma &\sim \Normal(0, TK)\\
\delta &= 10^{\eta + \zeta X_i}\\
\eta &\sim N(\mu_{\log \delta}, .25^2)\\
\zeta &\sim N(0., .125^2)
\end{align*}

\begin{figure}
\begin{center}
a)
\includegraphics[width=\textwidth]{hep_c-consistent0.pdf}
b)
\includegraphics[width=\textwidth]{hep_c-consistent1.pdf}
\end{center}
\caption{Caption TK, TK figure showing the results of a consistent fit
  assuming time stationarity with no priors on remission and
  excess-mortality, with priors on remission and excess-mortality that
  experts would agree to, and the results of a fit of prevalence only.
}
\label{hep_c-consistent}
\end{figure}

In the case of the data for North America, High Income in 1990, there is a
high-quality, nationally representative dataset from the NHANES study
that provides a prevalence age pattern with age groups only slightly
different from those needed for the GBD2010 study.  For this data,
visual inspection shows that the age integrating negative binomial
model for the prevalence rate suffices for estimating prevalence for
the GBD2010 age groups (Figure~\ref{hep_c-data}).  
This figure also shows the effects of varying the smoothing prior on
the piecewise constant Gaussian process.

In the case of other regions, or even this region for the 2005 time period, nationally representative data
from an NHANES-like study is not available, and the age integrating negative binomial model provides a way
to combine the noisy data.  Here, however, the expert priors have even
more of an effect on the results:
\begin{figure}
\begin{center}
\includegraphics[width=\textwidth]{hep_c-heterogeneity.pdf}
\end{center}
\caption{Caption TK, TK figure showing the results varying expert priors on heterogeneity and knot spacing.
}
\label{hep_c-heterogeneity}
\end{figure}

TK figure showing different results for a noisier region that NAHI,
possibly by varying smoothness prior and heterogeneity prior.  In the
caption, perhaps, a comparison of the dic values, showing which model
is prefered according to this metric.

Posterior predictive checks as an alternative approach to telling if
the model is doing a reasonable job.  TK Particular focus on the bias
effect coefficient in the 2005 NAHI estimate.

The priors on $\alpha_{r}, \alpha_{sex}, \alpha_{time}, \beta, and
\gamma_{r,s,t,a}$ are sufficiently uninformative that changing them
all by TK changes the results by less than TK.  The uninformativity of
these priors does affect computational efficiency, however, and TK
quantification of how much time is needed to fit the less informative
models.
