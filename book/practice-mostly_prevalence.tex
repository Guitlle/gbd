\section{Modeling diseases with mostly prevalence data}

A surprising headline of the first GBD study was the unexpectedly
large burden of mental disorders like depression and
schizophrenia. The epidemiology of these conditions, as well as many
other mental and neurological disorders, are known primarily from
prevalence studies. This chapter considers in detail the setting where
prevalence data is the primary source of input data.

Several gynecological disorders had systematic reviews which resulted
in only a small amount of prevalence data, and no other measurements
of epidemiological rates like incidence, remission, or mortality.  (TK
really no studies on remission?)

TK A lot of prevalence, expert priors on everything else

TK Prevalence plus a few data points on other things

TK Prevalence data which clearly violates time equilibrium assumption

TK Birth prevalence data only – no need for fancy statistical model

\subsection{PMS}
TK Figure showing model 16314

Since there is only prevalence data available, and even that is sparse
and extremely noisy, choices made in the modeling process will have a
significant effect on the resulting estimates.  This highlights the
importance of making it clear what the assumptions are, and how
sensitive the results are to these assumptions.

The data plot in Figure~\ref{TK} shows just how sever the situation
is.  If you come up with a number between 0 and 1, I can find a study
where the prevalence is within 5 per 100.

The prudent solution in this situation would be to give up, and not
even produce estimates of PMS prevalence, let alone incidence and
duration.  For a researcher who really must know the answer, recourse
would be to field a study designed to measure the quantity of interest
in the population of interest.  Since all of the wildly varying data
collected in systematic review shows that the prevalence is at least
$<<d['pms.json|dexy']['min_rate_per_100']>>$ per $100$.

However, I cannot opt out of estimating and I cannot wait for a new
definitive study to be conducted (although I can use the results of
the systematic review to decide which studies are highest priority in
the future).  So model-based estimates are my only option.

The simplifying assumptions that experts in these diseases have agreed
to are the following: there no incidence or prevalence of disease
before age 15 or after age 50, there is no excess mortality, and there
is no remission before age 40.  These assumptions correspond breaks
and midpoints of the age groups to be estimated in the GBD 2010 study,
which include groups for 10-15, 15-20, 20-25, 25-35, 35-45, and 45-55.

With these restrictions to the age patterns in place, the consistent
fit of all of the world's pooled data is the following:
This model is sensitive to the coarse knot selection in the age
pattern model, which must be chosen finely enough to allow consistent
fits that respect the expert priors on when age-specific rates are
non-zero.

TK figure comparing coarse, fine, and extra fine age mesh.

TK figure showing the posterior predictive distributions for all of
the worlds pooled data, and says what the over-dispersion parameter
for prevalence data is in median and HPD.

This provides a reasonable starting point, but TK discussion of
over-dispersion posterior and posterior predictive checks of data.  We
can potentially do better by using covariates to explain some of the
variation seen in the data.  TK discussion of relevant study-level
covariates.  TK discussion of the existance or lack thereof of
appropriate country-level covariates.  health system access?

TK figure comparing posterior predictive distribution for certain data
in model with no covariates, with study level covariates, and with
study + country level covariates.

TK figure showing how these choices affect the age pattern mean and
uncertainty.

TK discussion of a demonstration of how including these covariates
results in less dispersed posterior predictive distributions, and a
comparison of the posterior values of the over-dispersion terms in
both.  Also appropriate here to look at the BIC and DIC values.  AIC
is not appropriate for reasons to be stated.

Discussion of producing estimates that differ by region and  by time
using an empirical bayes approach, with the world estimate as the
empirical prior.

TK discussion of the possibility of an acceptible result, although it
is important to investigate the effects of smoothing priors,
heterogeneity priors, level bounds.

\subsection{Cannabis Dependence, also use}
14390 15303 

16153 16160 

\subsection{Bipolar}
The systematic review for epidemiological rates related to bipolar
disorder came up with TK rows of prevalence data, and TK rows of
standardized mortality ratio (SMR) data.  Since there is so much more
prevalence data than SMR data, I have pooled all of the rows of SMR
data across studies and applied them assuming that SMR for bipolar
does not vary by region or time.  There may be sufficient data to assume
that it \emph{does} vary by sex, however. 

The approach I followed in this setting is one that has come up quite
frequently.  Generate an empirical prior on prevalence, by pooling all
of the world's data in a model with effects for region, sex, and
time, and then generate region/sex/time specific posterior estimates
by applying the consistent model for the appropriate subset of the
prevalence data, together with all the rows of SMR data, and with a
minimal, defensible set of expert assumptions to fill in the remaining
model flexibility, in the case, the assumption that the remission rate
is at most .05 per person-year.

\subsection{Depression outside of North America}
12539 
16152
 The prevalence data, on the
other hand, does cover the majority of the 21 regions that partition
the world in the GBD2010 study, albeit quite non-uniformly.  There are
over 100 rows of data about each of North American High Income and
Western Europe, while there are less than 10 rows of data about Latin
America, Center; Latin America, Southern; and Sub-Saharan Africa,
Southern; and no rows of data at all about 4 other regions.

\subsection{Hepatitis C}
Disease with strong cohort effects, like hepatitis C in North America,
can violate the assumptions from Section~\ref{TK} enough to
necessitate a special approach.  That is what is investigated in this
section.

The following figure shows the problem:  there is a peak in prevalence
for age group TK, yet there is a strong belief among experts that
there is not remission or excess-mortality during this age group to
drive the decrease in older age groups.  The story from experts, which
I have no reason to disbelieve, is that there is a peak in prevalence
cause by transmission through heightened interveneous drug use during
the early 1970s.  This cohort, which is moving through the population
as the 20-30 year olds of the 1970s age into the 40-50 year olds of
the 1990s, violates the ``time stationarity'' assumption from
Section~\ref{TK} so much that any incidence rates generated from this
prevalence data would be completely unbelievable.  Even fitting the
prevalence data to be consistent with expert derived bounds on
remission and mortality levels is not possible; see Figure~\ref{TK}.

TK figure showing the results of a consistent fit assuming time
stationarity with no priors on remission and excess-mortality, with
priors on remission and excess-mortality that experts would agree to,
and the results of a fit of prevalence only.

In the case of the data for North America High Income, there is a
high-quality, nationally representative dataset from the NHANES study
that provides a prevalence age pattern with age groups only slightly
different from those needed for the GBD2010 study.  The results of the
model provide a fine demonstration of the effects of the smoothness
prior as well.

In the case of other regions, where nationally representative data
from an NHANES-like study is not available, the model provides a way
to combine the noisy data.  Here, however, the expert priors do have
more of an effect on the results:

TK figure showing different results for a noisier region that NAHI,
possibly by varying smoothness prior and heterogeneity prior.  In the
caption, perhaps, a comparison of the dic values, showing which model
is prefered according to this metric.

Posterior predictive checks as an alternative approach to telling if
the model is doing a reasonable job.

The model reduces to the following, for a row of data collected in
systematic review, with prevalence $p_i$ and effective sample size $n_i$ for
age group $(a_{0_i}, a_{1_i})$, the rate model, age pattern model,
and age group model from Chapter~\ref{TK} are used, together with
covariates $(t_i,s_i,x_i)$ for time, sex, and ``bias'', but without any requirement to
fit together consistently with incidence, remission, or
excess-mortality, in the following mixed effects negative binomial
regression:
\begin{align*}
p_i n_i &\sim \NegativeBinomial(\pi_i n_i, \delta)\\
\pi_i &= \int _{a=a_{0_i}} ^{a=a_{1_i}} \boldpi(a) dw(a)\\
\boldpi &\sim \PLGP(\mu_\boldpi, \calC)\\
\mu_\boldpi &= \exp\left\{ t_i \alpha_t + s_i \alpha_s + x_i \beta + \boldgamma(a) \right\}\\
\alpha_t, \alpha_s &\sim \Normal(0, TK)\\
\beta &\sim \Normal(0, TK)\\
\boldgamma(a) &\sim \text{PiecewiseLinearSpline}(0, TK)
\end{align*}

The hyper-priors on $\alpha_t, \alpha_s, \beta, and \boldgamma$ are
sufficiently uninformative that changing them all by TK changes the
results by less than TK.  The uninformativity of these priors does
affect computational efficiency, however, and TK quantification of how
much time is needed to fit the less informative models.

TK Particular focus on the bias effect coefficient in the 2005 NAHI estimate.
