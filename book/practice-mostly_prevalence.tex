\section{Modeling diseases with mostly prevalence data}

A surprising headline of the first GBD study was the unexpectedly
large burden of mental disorders like depression and
schizophrenia. The epidemiology of these conditions, as well as many
other mental and neurological disorders, are known primarily from
prevalence studies. This chapter considers in detail the setting where
prevalence data is the primary source of input data.

Several gynecological disorders had systematic reviews which resulted
in only a small amount of prevalence data, and no other measurements
of epidemiological rates like incidence, remission, or mortality.  (TK
really no studies on remission?)

TK A lot of prevalence, expert priors on everything else

TK Prevalence plus a few data points on other things

TK Prevalence data which clearly violates time equilibrium assumption

TK Birth prevalence data only – no need for fancy statistical model

\subsection{PMS}
TK Figure showing model 16314

Since there is only prevalence data available, and even that is sparse
and extremely noisy, choices made in the modeling process will have a
significant effect on the resulting estimates.  This highlights the
importance of making it clear what the assumptions are, and how
sensitive the results are to these assumptions.

The data plot in Figure~\ref{TK} shows just how sever the situation
is.  If you come up with a number between 0 and 1, I can find a study
where the prevalence is within 5 per 100.

The prudent solution in this situation would be to give up, and not
even produce estimates of PMS prevalence, let alone incidence and
duration.  For a researcher who really must know the answer, recourse
would be to field a study designed to measure the quantity of interest
in the population of interest.  Since all of the wildly varying data
collected in systematic review shows that the prevalence is at least
$<<d['pms.json|dexy']['min_rate_per_100']>>$ per $100$.

However, I cannot opt out of estimating and I cannot wait for a new
definitive study to be conducted (although I can use the results of
the systematic review to decide which studies are highest priority in
the future).  So model-based estimates are my only option.

The simplifying assumptions that experts in these diseases have agreed
to are the following: there no incidence or prevalence of disease
before age 15 or after age 50, there is no excess mortality, and there
is no remission before age 40.  These assumptions correspond breaks
and midpoints of the age groups to be estimated in the GBD 2010 study,
which include groups for 10-15, 15-20, 20-25, 25-35, 35-45, and 45-55.

With these restrictions to the age patterns in place, the consistent
fit of all of the world's pooled data is the following:
This model is sensitive to the coarse knot selection in the age
pattern model, which must be chosen finely enough to allow consistent
fits that respect the expert priors on when age-specific rates are
non-zero.

TK figure comparing coarse, fine, and extra fine age mesh.

TK figure showing the posterior predictive distributions for all of
the worlds pooled data, and says what the over-dispersion parameter
for prevalence data is in median and HPD.

This provides a reasonable starting point, but TK discussion of
over-dispersion posterior and posterior predictive checks of data.  We
can potentially do better by using covariates to explain some of the
variation seen in the data.  TK discussion of relevant study-level
covariates.  TK discussion of the existance or lack thereof of
appropriate country-level covariates.  health system access?

TK figure comparing posterior predictive distribution for certain data
in model with no covariates, with study level covariates, and with
study + country level covariates.

TK figure showing how these choices affect the age pattern mean and
uncertainty.

TK discussion of a demonstration of how including these covariates
results in less dispersed posterior predictive distributions, and a
comparison of the posterior values of the over-dispersion terms in
both.  Also appropriate here to look at the BIC and DIC values.  AIC
is not appropriate for reasons to be stated.

Discussion of producing estimates that differ by region and  by time
using an empirical bayes approach, with the world estimate as the
empirical prior.

TK discussion of the possibility of an acceptible result, although it
is important to investigate the effects of smoothing priors,
heterogeneity priors, level bounds.

\subsection{Cannabis Dependence, also use}
14390 15303 

16153 16160 

\subsection{Bipolar}
14661 
16151 

\subsection{Depression outside of North America}
12539 
16152
