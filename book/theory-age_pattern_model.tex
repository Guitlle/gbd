\section{Age pattern models}

In this section, I develop the theoretical decisions behind the model for age patterns used below.

The design criteria for age pattern modeling in this Bayesian
statistical framework is to come up with a prior distribution on
non-negative functions of age, i.e. a probability density
$\dens(f\given\theta)$ that is defined on functions
$f:[0,\infty]\rightarrow[0,\infty]$.  These functions will be used as
the age patterns for flows in the systems dynamics model from section
TK.

There are two popular approaches for modeling smooth functions
statistically, splines and Gaussian processes. TK Background on
splines.

TK background on Gaussian processes and the Matern covariance
function.

I have drawn on elements of both of these approaches, together with
the implications of the simplifying assumptions made in Chap TK about
the piecewise constant nature of the rates in the system of
differential equations.  For incidence, remission, and excess
mortality, I will develop a prior distribution that could be called a
``piecewise constant spline'', with knots chosen based on expert
knowledge of the epidemiology of the disease under consideration.
Examples TK, ADHA, Anxiety, Cannabis, Dementia.  For prevalence, I
have an analogous ``piecewise linear spline'' designed for reasons of
computational efficiency; the prevalence of the system is produced a
the solution to a series of differential equations, and it has high
computational cost.  To make the computation managable, I solve the
differential equations only for knots of the spline and then use
linear interpolation to fill in prevalence values between the knots.

This semi-parametric functional form is combined with the Matern
covariance function, inspired from Gaussian process theory, by using a
variance/covariance matrix derived from the Matern covariance function
to a Multivariate Normal prior on the knows of the piecewise constant
spline.

The following specification makes this precise:
\begin{align*}
TK
\end{align*}

Here $\rho$ is an hyper-prior, usually chosen from $3$
possibilities based on ``expert input''.  Unfortunately, experts do
not have an intuitive understanding of $\rho$.  Figure TK is intended
to help build intuition.

TK Figure on slightly, moderately, and very smooth priors.

Tk Special treatment of the prevalence prior is still on control
points of spline, but age pattern is from linear interpolation, since
the differential equations are more accurately approximated by this
than by a (continuous) piecewise constant function.

TK Special treatment for other derived values, since they are
combinations of continuous piecewise linear and piecewise constant
functions, they would end up being non-continuous piecewise linear
functions.  Since this is confusing, they are instead modeling as
piecewise constant functions, but with knots at the midpoints of the
knots for incidence, remission, and mortality.

TK Pitfalls of combining the smoothness constraints (?)

TK Additional Priors on age patterns: level bounds, level values,
increasing, decreasing.
\begin{verbatim}
1)Detailed explanation of all expert priors (100-200 pages)
a)Smoothness Prior
i)theory
ii)analogy
iii)examples with real data
b)heterogeneity
i)theory
ii)emphasis on unexplained heterogeneity
iii)examples with real data
c)level value
i)best practices in setting level values
(1)working with overly broad age intervals
ii)emphasize that this is a very strong prior
d)level bounds
i)analogy of MCMC to drunken mountaineering, and level value to a
fence
ii)examples with real data, including improved fits and improved
computation time
e)increasing and decreasing priors
i)theory
ii)discussion of the age range parameters
iii)examples of effects
\end{verbatim}


TK Future work: Unimodal priors [ref TK].  Data driven hyper-priors
for GP.  More advanced integration to lift piecewise constant
requirement on flows.

Extend to time/age models.

Still to write: covariates ``random effects'', macroeconomic derived
covariates, interpreted as shifting the level of the age pattern.
Future work change age pattern itself (``cokreiging'') also relax
linear assumption of shift.  This will be computationally challenging.

