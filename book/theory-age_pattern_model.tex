\section{Age pattern models}

In this section, I develop the theoretical decisions behind the
statistical model for age patterns used below.

The design criteria for age pattern modeling in this Bayesian
statistical framework is to come up with a prior distribution on
non-negative functions of age, i.e. a probability density
$\dens(f\given\theta)$ that is defined on functions
$f:[0,\infty]\rightarrow[0,\infty]$.  These functions will be used as
the age patterns for flows in the systems dynamics model from
Chapter~\ref{chapter:system-dynamics}.

There are two popular approaches for modeling smooth functions
statistically, splines and Gaussian processes. TK Background on
splines.

TK background on Gaussian processes and the Matern covariance
function, elaborating on the following facts: Wikipedia provides a fine background:
\begin{quote}
In probability theory and statistics] a \emph{Gaussian process} is a
  stochastic process whose realizations consist of random variable
  associated with every point in a range of times (or of space) such
  that each such random variable has a normal distribution. Moreover,
  every finite collection of those random variables has a multivariate
  normal distribution.

Gaussian processes are important in statistical modeling because
of properties inherited from the normal distribution. For example, if
a random process is modeled as a Gaussian process, the distributions
of various derived quantities can be obtained explicitly. Such
quantities include: the average value of the process over a range of
times; the error in estimating the average using sample values at a
small set of times.

A Gaussian process is a stochastic process $\{X_t : t \in T\}$ for
which any finite linear combination of samples will be normally
distributed (or, more generally, any linear functional applied to the
sample function $X_t$ will give a normally distributed result).

Alternatively, a process is Gaussian if and only if for every finite
set of indices $t_{1,\ldots,k}$ in the index set $T$,
\[
 \vec{\mathbf{X}}_{t_1, \ldots, t_k} = (\mathbf{X}_{t_1}, \ldots, \mathbf{X}_{t_k})
\]

is a vector-valued Gaussian
random variable. Using characteristic functions of random variables, the
Gaussian property can be formulated as
follows: $\{X_t : t \in T\}$
is Gaussian if and only if, for every finite set of indices
$t_1, \ldots, t_k$, there are reals
$\sigma_{l j}$ with $\sigma_{ii} > 0$ and reals
$\mu_j$ such that

\[
\operatorname{E}\left(\exp\left(i \ \sum_{\ell=1}^k t_\ell
\ \mathbf{X}_{t_\ell}\right)\right) = \exp \left(-\frac{1}{2} \,
\sum_{\ell, j} \sigma_{\ell j} t_\ell t_j + i \sum_\ell \mu_\ell
t_\ell\right).
\]

The numbers $\sigma_{lj}$ and $\mu_j$ can be shown to be the
covariances and means of the variables in
the process \cite{WP:GP}.

The ``Mat\'{e}rn covariance'' (named after the Swedish forestry
statistician Bertil Mat\'{e}rn) is a covariance function used in
spatial statistics, geostatistics, machine learning, image analysis,
and other applications of multivariate statistical analysis on metric
spaces. It is commonly used to define the statistical covariance
between measurements made at two points that are $d$ units distant
from each other. Since the covariance only depends on distances
between points, it is stationary process. If the distance is Euclidean
distance, the Mat\'{e}rn covariance is also isotropic.

The Mat\'{e}rn covariance between two points separated by $d$ distance
units is given by
\[
\calC(d) =
\sigma^2\frac{1}{\Gamma(\nu)2^{\nu-1}}\Bigg(2\sqrt{\nu}\frac{d}{\rho}\Bigg)^\nu
K_\nu\Bigg(2\sqrt{\nu}\frac{d}{\rho}\Bigg),
\]

where $\Gamma$ is the gamma function, $K_\nu$ is the modified Bessel
function of the second kind, and $\rho$ and $\nu$ are non-negative
parameters of the covariance.

A Gaussian process with Mat\'{e}rn covariance has sample paths that are
$\lceil \nu-1 \rceil$ times differentiable. As $\nu\rightarrow\infty$,
the Mat\'{e}rn covariance converges to the squared exponential covariance
function
\[
C(d) = \sigma^2\exp(-d^2/\rho^2). \,
\]
When $\nu = 0.5$, the Mat\'{e}rn covariance is identical to the
exponential covariance function \cite{WP:Matern}.
\end{quote}

Following the conventions of the Mat\'{e}rn implementation in the PyMC GP
Package, I call $\sigma$ the amplitude, $\rho$ the scale, and $\nu$
the degree of differentiability.

I have drawn on elements of both of these approaches, together with
the implications of the simplifying assumptions made in
Chapter~\ref{theory-forward_sim-compartmental_model-simplying_assumptions}
about the piecewise constant nature of the rates in the system of
differential equations.  For incidence, remission, and excess
mortality, I will develop a prior distribution that could be called a
``piecewise constant spline'', with knots chosen based on expert
knowledge of the epidemiology of the disease under consideration.

Examples TK, ADHA, Anxiety, Cannabis, Dementia.

For prevalence, I have an analogous ``piecewise linear spline''
designed for reasons of computational efficiency; the prevalence of
the system is produced a the solution to a series of differential
equations, and it has high computational cost.  To make the
computation managable, I solve the differential equations only for
knots of the spline and then use linear interpolation to fill in
prevalence values between the knots.

This semi-parametric functional form is combined with the Matern
covariance function, inspired from Gaussian process theory, by using a
variance/covariance matrix derived from the Matern covariance function
to a Multivariate Normal prior on the knows of the piecewise constant
spline.

The following specification makes this precise:
\begin{align*}
f(a) \given \mu,\rho &\sim
\begin{cases} \GP(\mu, \calC) &\qquad\text{if } f(a) \geq 0 \text{ for all } a \geq 0,\\
0 &\qquad\text{otherwise;}\\
\end{cases}\\
\log \mu &\sim \Normal(0, 2^2);\\
\calC &= \Matern(\nu, \sigma, \rho);\\
\nu &= 2.
\end{align*}

Here $\rho$ and $\sigma$ are an hyper-priors.  In practice, $\rho$ is
usually chosen from $3$ possibilities corresponding to ``slightly'',
``moderately'', and ``very'' smooth, based on expert judgement.
Unfortunately, experts do not have an intuitive understanding of
$\rho$. Figure~\ref{theory-age_pattern_model-smoothness_priors} is
intended to help build intuition. The autocovariance plot shows how a
slight smoothing prior of $\rho=50$ leads to a prior where the value
of $f$ at age $a$ explains $50\%$ of the variation in $f$ at age
$a+TK$.  

\begin{figure}[h]
\begin{center}
\includegraphics[width=\textwidth]{smoothness_priors.pdf}
\includegraphics[width=\textwidth]{smoothness_covariance.pdf}
\end{center}
\caption{TK Figure on slightly, moderately, and very smooth priors.}
\label{theory-age_pattern_model-smoothness_priors}
\end{figure}

Tk Special treatment of the prevalence prior is still on control
points of spline, but age pattern is from linear interpolation, since
the differential equations are more accurately approximated by this
than by a (continuous) piecewise constant function.

TK Special treatment for other derived values, since they are
combinations of continuous piecewise linear and piecewise constant
functions, they would end up being non-continuous piecewise linear
functions.  Since this is confusing, they are instead modeling as
piecewise constant functions, but with knots at the midpoints of the
knots for incidence, remission, and mortality.

TK Pitfalls of combining the smoothness constraints (?)

TK Additional Priors on age patterns: level bounds, level values,
increasing, decreasing.
\begin{verbatim}
1)Detailed explanation of all expert priors (100-200 pages)
a)Smoothness Prior
i)theory
ii)analogy
iii)examples with real data
b)heterogeneity
i)theory
ii)emphasis on unexplained heterogeneity
iii)examples with real data
c)level value
i)best practices in setting level values
(1)working with overly broad age intervals
ii)emphasize that this is a very strong prior
d)level bounds
i)analogy of MCMC to drunken mountaineering, and level value to a
fence
ii)examples with real data, including improved fits and improved
computation time
e)increasing and decreasing priors
i)theory
ii)discussion of the age range parameters
iii)examples of effects
\end{verbatim}


TK Future work: Unimodal priors [ref TK].  Data driven hyper-priors
for GP.  More advanced integration to lift piecewise constant
requirement on flows.

Extend to time/age models.

Still to write: covariates ``random effects'', macroeconomic derived
covariates, interpreted as shifting the level of the age pattern.
Future work change age pattern itself (``cokreiging'') also relax
linear assumption of shift.  This will be computationally challenging.

