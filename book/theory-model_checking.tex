\section{Model Checking and Model Selection}
\label{theory-model_checking}

In this section, I explore several approaches for assessing the
quality of estimates produced by fitting the integrative model.  I
begin with a qualitative check based on overlaying model predictions
and age interval data and a similar approach based on comparing
posterior predictive distributions to data.  I also consider the
feasibility of using hold-out cross validation and some variants,
which I feel is not appropriate in most of my applications.  I also
consider so-called ``information criteria'', quantified as the
Bayesian information criteria (BIC) and divergence information
criteria (DIC), which are not absolute tests of model fitness, but can
be used for comparing two different models of the same data.  This
leads me to a discussion of sensitivity analysis, where I set out
some principles for determining how sensitive the model is to the expert
priors.  I conclude by contrasting these approaches to each other, and
also to simulation study, a parallel approach to confirming that the
model is producing trustworthy results that is pursued in
Chapter~\ref{TK}.

\subsection{Graphical model checking}
Simply looking at the estimates produced by the model together with
the data that they derive from is an important first step in model
checking.  You could call this checking for face validity.  Since the
age pattern is so important to the integrative model, I like to look
at the estimates as a function of age, and since there is so much
heterogeneity in data, I like to stratify my graphical model checking
by time period, sex, and region.  An example of the sort of results
that this approach produces appear in Figure~\ref{graphical-model-checking}.

\begin{figure}
\begin{center}
TK Graphic check of model fit
\end{center}
\caption{Graphical model checking by comparing the model estimate to
  the input data as a function of age, for different time periods,
  sexes, and regions.  It is clear that the model estimates are
  related to the data, but it would be hard to decide between these
  estimates and slightly different ones based on this sort of check
  alone.}
\label{graphical-model-checking}
\end{figure}

During this sort of qualitative assessment of model fit, it is also
appropriate to look at the direction and magnitude of the effect
coefficients for any covariates that were included in the model.  Do
the model parameter estimates show that the direction of the effect is
plausible?  This is another face-validity check that is sometimes
called "the laugh test".  When producing initial estimates for a new
model, this level of qualitative analysis will be sufficient to determine
that something has gone wrong when coding the data, or that an expert
prior has been set incorrectly, or that the numerical algorithm
responsible for fitting the model parameters has failed to converge.

As a first attempt to go beyond the qualitative comparison of age
pattern estimates to model data, I like to look more closely at the
posterior predicted distribution for each row of input data.  This is
a form of posterior predictive check \ref{TK}, which provides more
information about why the model has produced the estimates it has produced.
These plots show the rate value and uncertainty for each row of data
collected in systematic review, together with the posterior
distribution for the predicted value of the rate for this row.  In an
excellent model fit, 95\% of the input data will fall within the 
predicted 95\% HPD interval, although it is not required that this be
the case.    This is a good way to identify systematic review data to
recheck, since it provides a model-based approach to identifying
outliers, which could be cause by data coding errors, or by
incorrectly including studies which should have been rejected during
systematic review.  Another sign of a model fitting well is if the residuals
are balanced in sign, with roughly half the estimates above the
observed values and half below.  Although this is not required either,
it is reassuring when it is the case.


\subsection{Hold-out cross validation}
A popular approach of assessing the quality of predictive models is to
use hold-out cross validation, where a random subset of data is
withheld while fitting the model, and then the fit model is used to
predict the values of the held-out data.  This gives an
``out-of-sample'' measure of the model's predictive accuracy.  TK more
formal version of this, possibly with a figure. There are two
difficulty in using this approach in my setting however.  The data is
too sparse and too noisy.

The sparseness problem is clear: when there is an overwhelming supply of
data, the results of fitting a model with all of it is very similar to
fitting a model with 75\% of it.  But when there is not very much data
to begin with, the results of fitting a model with only a subset of it
can lead to huge changes.  On the one hand, this is useful
information: the model is extrememly sensitive to the data available.
On the other hand, it can be misleading because the marginal value of
additional data is very high in this setting, and we are interested in
the predictive accuracy for the model when we have all the data we
have, not when we have 75\% of it.  This problem can be addressed by
choosing carefully how much data to hold out.  The standard used in
the machine learning groups of the computer science departments (where
this technique is the coin of the realm \ref{TK}) is to hold out 20 or
25\%.  But in the statistics department, it is common to use all but
one data point, in so-called leave-one-out cross validation.
Somewhere in between could allow me to use hold-out cross validation
for the disease model, but finding that point is difficult, and any
choice would appear arbitrary.

The problem that noisy data causes is more subtle.  It really gets to
the question of what I am trying to predict.  I am not trying to
do the impossible.  I am not trying to predict if a fair coin will
come up heads or tails, and I am not trying to predict what rate will
come up in a small study in a heterogenous population.  But that is
what hold-out cross validation is judged on (and posterior predictive
checks, too, for that matter, but qualitatively and ``in-sample'').
If the data is extremely noisy (as it often is), the predictive
accuracy measured by hold-out cross validation will be quite low, but
it is not the model's fault, it's all due to the data.  This is even
more pronounced when the data is biased.  A consistent model that
knows prevalence is often overestimated and incidence is often
underestimated could correct for both of these things, yet be judged
worse in predictive accuracy by a hold-out cross validation.

\subsection{Information criteria}
The theory of Bayesian inference provides several additional
quantitative measures of model fit, which I have also considered but
found to be of limited utility, such as the Bayesian information
criterion, the divergence information criterion, and the Bayes factor.

Wikipedia
\url{http://en.wikipedia.org/wiki/Bayesian_information_criterion}
\begin{quote}
The BIC was developed by Gideon E. Schwarz, who gave a Bayesian
argument for adopting it \ref{doi:10.1214/aos/1176344136}.

BIC equals $-2\log L + k\log n$ where $n$ is the number of observations,
$k$ is the number of free parameters to be estimated, and $L$ the
maximized value of the likelihood function for the estimated model.

lower BIC implies either fewer explanatory variables, better fit, or
both. 

It is important to keep in mind that the BIC can be used to compare
estimated models only when the numerical values of the dependent
variable are identical for all estimates being compared. The models
being compared need not be nested, unlike the case when models are
being compared using an F or likelihood ratio test.
\end{quote}

In my experience the BIC is quite sensitive to numerical algorithms,
and since it takes only the maximum likelihood of the model into
account, it seems ill suited for models where a good fit is expected
to have large uncertainty.

The DIC, Wikipedia
\url{http://en.wikipedia.org/wiki/Deviance_information_criterion}

\begin{quote}
The deviance information criterion (DIC) is a hierarchical modeling
generalization of the AIC (Akaike information criterion) and BIC
(Bayesian information criterion, also known as the Schwarz
criterion). It is particularly useful in Bayesian model selection
problems where the posterior distributions of the models have been
obtained by Markov chain Monte Carlo (MCMC) simulation. Like AIC and
BIC it is an asymptotic approximation as the sample size becomes
large. It is only valid when the posterior distribution is
approximately multivariate normal.

Define the deviance as $D(\theta)=-2 \log(p(y|\theta))+C$, where $y$
are the data, $\theta$ are the unknown parameters of the model and
$p(y|\theta)$ is the likelihood function. $C$ is a constant that
cancels out in all calculations that compare different models, and
which therefore does not need to be known.

The expectation $\bar{D}=\mathbf{E}^\theta[D(\theta)]$ is a measure of
how well the model fits the data; the larger this is, the worse the
fit.

The effective number of parameters of the model is computed as
$p_D=\bar{D}-D(\bar{\theta})$, where $\bar{\theta}$ is the expectation of
$\theta$. The larger this is, the easier it is for the model to fit
the data.

The deviance information criterion is calculated as
\[
    \mathit{DIC} = p_D+\bar{D}.
\]
The idea is that models with smaller DIC should be preferred to models
with larger DIC. Models are penalized both by the value of $\bar{D}$,
which favors a good fit, but also (in common with AIC and BIC) by the
effective number of parameters $p_D$. Since $\bar D$ will decrease as
the number of parameters in a model increases, the $p_D$ term
compensates for this effect by favoring models with a smaller number
of parameters.

The advantage of DIC over other criteria in the case of Bayesian model
selection is that the DIC is easily calculated from the samples
generated by a Markov chain Monte Carlo simulation. AIC and BIC
require calculating the likelihood at its maximum over $\theta$, which
is not readily available from the MCMC simulation. But to calculate
DIC, simply compute $\bar{D}$ as the average of $D(\theta)$ over the
samples of $\theta$, and $D(\bar{\theta})$ as the value of $D$ evaluated
at the average of the samples of $\theta$. Then the DIC follows
directly from these approximations.
\end{quote}

DIC seems more robust than BIC in my limited experience, although I am
not convinced of its utility either.

The Bayes factor is yet another approach to model comparison, which
can also be used of ``Bayesian hyptohesis testing''.

Wikipedia \url{http://en.wikipedia.org/wiki/Bayes_factor}:
\begin{quote}
The posterior probability of a model $M$, given data $D$, $\Pr(M|D)$, is
given by Bayes' theorem:
\[
    \Pr(M|D) = \frac{\Pr(D|M)\Pr(M)}{\Pr(D)}.
\]
The key data-dependent term $\Pr(D|M)$ is a likelihood, and is sometimes
called the evidence for model or hypothesis, $M$; evaluating it
correctly is the key to Bayesian model comparison. The evidence is
usually the normalizing constant or partition function of another
inference, namely the inference of the parameters of model $M$ given the
data $D$.

Given a model selection problem in which we have to choose between two
models, on the basis of observed data $D$, the plausibility of the two
different models $M1$ and $M2$, parametrised by model parameter vectors $\theta_1$
and $\theta_2$ is assessed by the Bayes factor K given by
\[
    K = \frac{\Pr(D|M_1)}{\Pr(D|M_2)} = \frac{\int
      \Pr(\theta_1|M_1)\Pr(D|\theta_1,M_1)\,d\theta_1} {\int
      \Pr(\theta_2|M_2)\Pr(D|\theta_2,M_2)\,d\theta_2 }. 
\]
where $\Pr(D|M_i)$ is called the marginal likelihood for model $i$.

If instead of the Bayes factor integral, the likelihood corresponding
to the Maximum likelihood estimate of the parameter for each model is
used, then the test becomes a classical likelihood-ratio
test.[citation needed] Unlike a likelihood-ratio test, this Bayesian
model comparison does not depend on any single set of parameters, as
it integrates over all parameter in each model (with respect to the
respective priors). However, an advantage of the use of Bayes factors
is that it automatically, and quite naturally, includes a penalty for
including too much model structure.[3] It thus guards against
overfitting. For models where an explicit version of the likelihood is
not available or too costly to evaluate numerically, approximate
Bayesian computation can be used for model selection in a Bayesian
framework.[4]

A value of $K > 1$ means that the data indicate that $M1$ is more strongly
supported by the data under consideration than $M2$. Note that classical
hypothesis testing gives one hypothesis (or model) preferred status
(the 'null hypothesis'), and only considers evidence against
it. Harold Jeffreys gave a scale for interpretation of $K$:[5]
$K   < 0$  Negative (supports M2); 
$    1:1 to 3:1 $
Barely worth mentioning;
$    3:1 to 10:1 $
    Substantial;
$    10:1 to 30:1 $
        Strong;
$        30:1 to 100:1 $
    Very strong;
$    >100:1 $
    Decisive.
\end{quote}

\subsection{Sensitivity Analysis}
The model as developed has a number of assumptions, some of which are
embedded in the system dynamics (like the assumption that disease
rates are not shifting over time), and some of which are elicited from
domain experts (like the smoothness of the incidence age pattern).
All of these assumptions are rightly the subject of scrutiny when
estimates based on the model are calculated and presented.  Some, such
as the time-stationarity assumption must be assessed outside of the
modeling framework for now.  But others, such as the choices of expert
priors, can be varied within the modeling framework to explore how
sensitive the results are to the subjective choices of the modeler and
domain expert.  For example, I find it worthwhile to check the effects
of changing the smoothness priors on the age pattern models for all
rates in the model, as well as the heterogeneity prior (prior mean of
the over-dispersion of the negative binomial model) for all rates for
which systematic review collected rate data.

A large portion of each application in the application sections in Chapter~\ref{TK}
considers the sensitivity of real models to these many choices.

\subsection{Comparison of alternatives}
Of the alternative approaches to model checking and model selection
discussed above, there is none that is clearly right or wrong.  All
seem to have room for improvement, yet all yield useful information
when the situation is right.  An important direction for future work
it to investigate the possibility of using one or more of the
quantitative methods above for ensemble modeling, where models with
different priors or covariates are combined in a mixture based on
in-sample or out-of-sample predictive accuracy.  This has proven
effective in many other settings, and would likely be helpful here,
too, especially in coming up with accurate quantification of
uncertainty.

Compare and contrast with reference to a specific example.

While these approaches are all important ways of assessing the model's
fit for a specific dataset, there is an alternative approach.  In
order to test the limits of the model when ground truth is known I can
use a simulation study.  This is the topic that I turn to next.

