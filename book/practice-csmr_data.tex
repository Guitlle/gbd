\section{Diseases with cause-specific mortality data}

TK diseases where there the excess-mortality for people with the
disease shows up as a certain ICD code on the death certificates.

TK diseases where excess-mortality exceeds $f_ICD$, but this is a lower
bound, and still the most relevant source of information.

Cause-specific mortality rates are not fundamental rates in the system
dynamics model from Chapter [ref TK]. They are measurements of a lower
bound on the prevalence times the excess-mortality rate. Nonetheless,
they can be informative information. TK figure showing how different
prevalence levels and excess-mortality levels can lead to the same
cause-specific mortality rate.

\section{Cirrhosis CSMR Data}
To validate the model and also to understand how the model performs in
a very common setting, this section investigates the results of
fitting the integrative systems model with data on the cause-specific
mortality rate of decompenstated cirrhosis.

We assume that all excess-mortality in decompenstated cirrohosis
cases will be coded as cause by cirrhosis on death certificates, so
that population level CSMR estimates are exactly applicable as
estimates for $pf$ in the system dynamics model.

Because we are using only this data, with nothing else, we assume that
there is zero remission, and that the (instantenous) excess-mortality
rate for cirrhosis cases is $2$ (corresponding to an average case
duration of $6$ months).

With these assumptions, the integrative systems model is fully
specified, albeit implicitly, in terms of $r$, $f$, and $pf$.  For any
age-patterns for these rates, there is either no solution to the
differential equations or a unique solution.

We used MCMC to obtain samples from the joint posterior distribution
on all model parameters, and marginal distributions of some of these
are shown in Figure~\ref{TK}.

