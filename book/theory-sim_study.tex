\subsection{Methods of Model Validation}
To show that DisMod produces accurate estimates from
sparse and noisy data, we conducted on a simulation experiment.
This approach tests the validity of the model in a setting where the estimates can
be compared to ``truth'' used in the simulation.  We varied experiment
parameters on data sparsity (30, 300, 3000 data points) and
data noise (coefficient of variation 2\%, 20\%, 200\%), to determine how
DisMod estimates perform on sparse and noisy data.

In the simulation experiment, \doublecheck{100} replicates were
generated for each experimental arm as follows: we generated ground
truth by choosing (age, sex, year)-specific incidence, remission, and
excess-mortality rates for each of 21 world regions to roughly follow typical
age patterns for diabetes, and then calculated consistent prevalence rates using
the region-specific all-cause mortality rate and the generic model of
disease progression from Eq.~\ref{eq:ode-soln} (Fig.~\ref{fig:sim-graphic}).  Then we simulated the
data collection and systematic literature review process by sampling a
collection of ages for each region, sex, and year and generating the
rate for each age by sampling from a normal distribution with mean
corresponding to the country-specific rate for that age and a variance
dictated by the coefficient of variation for each arm of the experiment.
We then estimated the consistent disease parameters
from this sparse, noisy dataset using the generic disease model.


\section{Results}
We validated our model on simulated data and found that with
\doublecheck{300} rows of data and coefficient of variation
\doublecheck{20\%}, the model-based imputation produces estimates with
Median Absolute Relative Error (MARE) \doublecheck{7.0 \%} with
interquartile range (\doublecheck{5}, \doublecheck{8}).  The coverage
probability (CP) for this level of noise and missingness was
\doublecheck{97.3 \%} with interquartile range (\doublecheck{97},
\doublecheck{99}).  Table~\ref{table:sim-results} and Figure~\ref{fig:sim-graphic} show more details
of simulation experiment results.  These results show that the generic
disease modeling approach is capable of recovering accurate estimates
in the presence of very sparse, very noisy data, although, like any
method, it is not without limitations.


%http://andrewjpage.com/index.php?/archives/43-Multirow-and-multicolumn-spanning-with-latex-tables.html
\begin{table}
\begin{center}
\begin{tabular}{|cc|cc|cc|}
\hline
$n$ (rows)      &       $cv$ (\%)            &       MARE (\%)       &       IQR     &       CP (\%) &       IQR\\
\hline
\hline
3000    &       2            &       1.4     &       (1, 2)  &      96.0     &       (95, 97)\\
3000    &       20           &       3.3     &       (2, 4)  &      93.1     &       (91, 96)\\
3000    &       200          &       63.1    &       (61, 68)       &
        25.2    &       (21, 27)\\
300     &       2            &       3.5     &       (2, 4)  &      97.3     &       (96, 99)\\
300     &       20           &       6.3     &       (5, 7)  &      97.3     &       (96, 99)\\
300     &       200          &       42.5    &       (32, 48)       &
        70.0    &       (61, 76)\\
30      &       2            &       11.8    &       (9, 14) &      91.6     &       (89, 97)\\
30      &       20           &       17.8    &       (13, 20)       &
        86.6    &       (81, 93)\\
30      &       200          &       59.4    &       (52, 77)       &
        43.6    &       (28, 53)\\
\hline
\end{tabular}
\end{center}
\caption{\doublecheck{Results of simulation experiment}.  For a range
  of dataset sizes (rows $n = 3000, 300, 30$) and varying noise levels
  (coefficient of variation $cv=2, 20, 200\%$), the median absolute
  relative error (MARE) was low and coverage probability (CP) was near
  the target value of 95\%.  The interquartile range (IQR) for these
  statistics was calculated over \doublecheck{100} independent
  replications of the experiment. Even for simulated data sets with
  very high noise and very high missingness, the model is able to
  produce estimates with accurate predictions of both mean and
  uncertainty.  However for noisy data with 200\% coefficient of
  variation or sparse data with less than .01\% of the
  age-region-sex-time points available, the model breaks down.}
\label{table:sim-results}
\end{table}

\begin{figure}
\includegraphics[width=\textwidth]{sim-graphic.eps}
\caption{Data from simulation study with $300$ rows and coefficient of
  variation $20\%$.  The dotted curves show the ground truth used in
  the simulation, the grey and black bars show the sparse, noisy
  samples of $300$ rows of input data, the solid stairstep curves show
  the estimates for a single region/sex/year, with the 95\%
  uncertainty interval for this estimate shaded in grey.  The black
  input data bars are from the region/sex/year estimated, while the
  grey input data bars are from other region/sex/years.  In this
  replicate of the experiment, the MARE was \doublecheck{7.13}\% and the
  CP was \doublecheck{96.2}\%.}
\label{fig:sim-graphic}
\end{figure}
