\section{Systematic review and meta-analysis}
TK History of systematic review and meta-analysis, from antiquity and
tracing its rise in prominanace in the medical sciences, togrether
with a justifi caion for its continuted practice, in terms of
summarizing the vast quantity of primary researech being
conducted. Distinguish systematic review from meta-analysis, following
the distinction made by PRISMA.

Wikipedia \url{http://en.wikipedia.org/wiki/Systematic_review}
\begin{quote} A systematic review is a literature review
  focused on a research question that tries to identify, appraise,
  select and synthesize all high quality research evidence relevant to
  that question. Systematic reviews of high-quality randomized
  controlled trials are crucial to evidence-based medicine.[1] An
  understanding of systematic reviews and how to implement them in
  practice is becoming mandatory for all professionals involved in the
  delivery of health care. Besides health interventions, systematic
  reviews may concern clinical tests, public health interventions,
  adverse effects, and economic evaluations.[2] Systematic reviews are
  not limited to medicine and are quite common in other sciences such
  as psychology, nursing, occupational therapy, physical therapy,
  educational research, sociology and business management.

A systematic review aims to provide an exhaustive summary of
literature relevant to a research question. The first step of a
systematic review is a thorough search of the literature for relevant
papers. The Methodology section of the review will list the databases
and citation indexes searched, such as Web of Science and PubMed, as
well as any individual journals. Next, the titles and the abstracts of
the identified articles are checked against pre-determined criteria
for eligibility and relevance. Each paper may be assigned an objective
assessment of methodological quality using the Jadad scale or similar
rating system.[3][4][5] Systematic reviews often, but not always, use
statistical techniques (meta-analysis) to combine results of the
eligible studies, or at least use scoring of the levels of evidence
depending on the methodology used. Systematic review is often applied
in the biomedical or healthcare context, but it can be applied in any
field of research. Groups like the Campbell Collaboration are
promoting the use of systematic reviews in policy-making beyond just
healthcare.

A systematic review uses an objective and transparent approach for
research synthesis, with the aim of minimizing bias. While many
systematic reviews are based on an explicit quantitative meta-analysis
of available data, there are also qualitative reviews which adhere to
the standards for gathering, analyzing and reporting evidence. The
EPPI-Centre has been influential in developing methods for combining
both qualitative and quantitative research in systematic reviews.[6]

Many healthcare journals now publish systematic reviews, but the
best-known[citation needed] source is The Cochrane Collaboration, a
group of over 28,000 specialists in health care who systematically
review randomised trials of the effects of prevention, treatments and
rehabilitation as well as health systems interventions. When
appropriate, they also include the results of other types of
research. Cochrane Reviews are published in The Cochrane Database of
Systematic Reviews section of The Cochrane Library. The 2010 impact
factor for The Cochrane Database of Systematic Reviews was 6.186, and
it was ranked 10th in the ``Medicine, General \& Internal'' category.[9]


\end{quote}

Wikipedia \url{http://en.wikipedia.org/wiki/Meta-analysis}
\begin{quote}
In statistics, a meta-analysis combines the results of several studies
that address a set of related research hypotheses. In its simplest
form, this is normally by identification of a common measure of effect
size, for which a weighted average might be the output of a
meta-analyses. Here the weighting might be related to sample sizes
within the individual studies. More generally there are other
differences between the studies that need to be allowed for, but the
general aim of a meta-analysis is to more powerfully estimate the true
``effect size'' as opposed to a smaller ``effect size'' derived in a
single study under a given single set of assumptions and conditions.

Meta-analyses are often, but not always, important components of a
systematic review procedure. Here it is convenient to follow the
terminology used by the Cochrane Collaboration,[1] and use
``meta-analysis'' to refer to statistical methods of combining
evidence, leaving other aspects of 'research synthesis' or 'evidence
synthesis', such as combining information from qualitative studies,
for the more general context of systematic reviews.

The term ``meta-analysis'' was coined by Gene V. Glass.[2]

The first meta-analysis was performed by Karl Pearson in 1904, in an
attempt to overcome the problem of reduced statistical power in
studies with small sample sizes; analyzing the results from a group of
studies can allow more accurate data analysis.[3][4] However, the
first meta-analysis of all conceptually identical experiments
concerning a particular research issue, and conducted by independent
researchers, has been identified as the 1940 book-length publication
Extra-sensory perception after sixty years, authored by Duke
University psychologists J. G. Pratt, J. B. Rhine, and associates.[5]
This encompassed a review of 145 reports on ESP experiments published
from 1882 to 1939, and included an estimate of the influence of
unpublished papers on the overall effect (the file-drawer
problem). Although meta-analysis is widely used in epidemiology and
evidence-based medicine today, a meta-analysis of a medical treatment
was not published until 1955. In the 1970s, more sophisticated
analytical techniques were introduced in educational research,
starting with the work of Gene V. Glass, Frank L. Schmidt and John
E. Hunter.

Gene V Glass was the first modern statistician to formalize the use of
meta-analysis, and is widely recognized as the modern founder of the
method. The online Oxford English Dictionary lists the first usage of
the term in the statistical sense as 1976 by Glass.[2][6] The
statistical theory surrounding meta-analysis was greatly advanced by
the work of Nambury S. Raju, Larry V. Hedges, Harris Cooper, Ingram
Olkin, John E. Hunter, Jacob Cohen, Thomas C. Chalmers, Robert
Rosenthal and Frank L. Schmidt.

\end{quote}

TK a discussion of the common cautions about comparing apples and
oranges in meta-analysis; quote from Cochrane Handbook
\url{http://www.cochrane-handbook.org/}.  But comparing apples and
oranges is exactly what we want to do sometimes, for example when
estimating population levels of nutritional risk factors, where
respondants are asked, ``how many servings of fresh fruits and
vegetables do you eat in a day?''

These concerns are not without grounds, however.  And the purpose of
this book is to develop a method for comparing the results of
descriptive epidemiological studies of disease prevalence, incidence,
remission, and mortality risk, which are focused on subpopulations
from varying age groups, sexes, regions, and time periods.  It is not
impossible, but it is not as easy as inverse variance weighting or
even random effects regression.

It is not without precedent, however.  The next section discusses the
legacy of ``generic disease modeling'', which my integrative approach
to descriptive epidemiological meta-analysis builds upon.


