\section{Systematic review and meta-analysis}
In 1904, the British military commissioned the statistician Karl
Pearson (for whom the correlation coefficient is named) to evaluate
the military's typhoid inoculation campaigns
\cite{Pearson_Report_1904}. Pearson obtained data on typhoid
inoculation and mortality from two studies, one from India and one
from South Africa, but determined that each study had too small a
sample size to permit a reliable analysis. To increase the sample
size, he decided to combine the data and thus embarked on the first
meta-analysis in public health.

Unfortunately, this landmark study concluded little. Pearson found the
data ``to be so heterogeneous and the results so irregular, that it
must be doubtful how much weight is to be attributed to the different
results''. Despite the inauspicious beginning, meta-analysis continued
to develop. In 1935, Ronald Fisher published the \emph{Design of
  Experiments} in which he made fundamental contributions to the
statistical theory underlying clinical trials. He also included a
meta-analysis in the textbook that combined data from multiple studies
to demonstrate that the effect of fertilizer varies over time and
geography \cite{O'Rourke_An_2007}. After Fisher, statisticians began
to develop more sophisticated techniques to combine studies by, for
instance, building models that allowed data from different analyses to
have different variances and thus to be averaged together with
different weights.

Systematic review has developed extensively since Pearson's time. The
sheer number of publications every year has forced researchers to
devise new ways to summarize and synthesize the torrent of data. From
1907, 3 years after the first meta-analysis, to 2007, the number of
scientific publications has exploded. Abstracts compiled by the
American Chemical Society have grown at 4.6\% a year over that
100-year period. Publications compiled by the American Mathematical
Society have grown at 5.9\% a year.  Publications in Compendex, a
database of engineering studies, has grown at 3.9\% a year
\cite{Larsen_Rate_2010}. PubMed, the largest database of biomedical
literature in the world, now contains more than 21 million citations
\cite{US_PubMed_2012}. Still, data are as heterogeneous and irregular
as ever. This challenge certainly remains for data in descriptive
epidemiology. Generic disease modeling provides a tool to get the most
out of that disparate data.

As the number of scientific publications grew, identifying various
sources to synthesize in a meta-analysis became a formidable task in
its own right. This challenge led to the formalization of the process
for identifying sources and the development of systematic review. The
Cochrane Collaboration defines systematic review as ``a review of a
clearly formulated question that uses systematic and explicit methods
to identify, select, and critically appraise relevant research, and to
collect and analyze data from the studies that are included in the
review''. Meta-analysis then is defined as ``the use of statistical
techniques in a systematic review to integrate the results of included
studies'' \cite{Green_Systematic_2005}.

The Cochrane Collaboration provides a prominent example of the utility
of systematic review and its current state. The collaboration is a
group of over 28,000 volunteers who review data from randomized
control trials of health interventions
\cite{Cochrane_Cochrane_2012}. In addition to the valuable information
they provide on the efficacy of a wide range of interventions, they
have provided a detailed handbook for conducting systematic
reviews.

The PRISMA group has also developed guidance for systematic reviews by
standardizing the steps involved in a modern approach to systematic
review \cite{Liberati_PRISMA_2009}. PRISMA divides the systematic
review process into four stages: Identification, Screening,
Eligibility and Included \cite{Green_Systematic_2005}.  In the
Identification stage, the reviewer finds citations for studies by
searching a database like PubMed and by contacting individual
researchers and institutions. The reviewer uses a specific set of
keywords for the database search in order to make that search
transparent and replicable. In the Screening stage, the reviewer
removes duplicated and unusable data. In Eligibility stage, the
reviewer excludes articles that do not match the explicit criteria for
inclusion in the study. For instance, some systematic reviews in
epidemiology only include evidence from randomized control trials and
exclude observational data. In the final Included stage, the reviewer
finalizes the studies used for the systematic review.

The Global Burden of Disease Project has undertaken this process for a
number of diseases. The methodological challenge is then to take the
resulting data as input to generate consistent estimates of
epidemiological parameters like incidence and remission.

Meta-analyses rely critically on the systematic review procedure. Here
it is convenient to follow the terminology used by the Cochrane
Collaboration and PRISMA, and use ``meta-analysis'' to refer to
statistical methods of combining evidence.  This provides a clear
separation of systematic review and meta-analysis, and also divides
meta-analysis from non-statistical approaches to of ``research
synthesis'' or ``evidence synthesis'', such as combining information
from qualitative studies.

Meta-analysis combines the results of several studies that address a
set of related research hypotheses. In its simplest form, this
proceeds by identification of a common measure of interest in all studies, for
which a weighted average might be the output of a meta-analyses. Here
the weighting might be related to sample sizes within the individual
studies.

By far the most common use of meta-analytic techniques is to estimate
the ``effect size'' of an intervention.  By pooling all studies of the
intervention effect, the meta-analysis attempts provide a more precise
estimate of effect size than those found in any individual single
study.

The Cochrane guidelines caution not to compare studies with very
different outcome measures of effect or very different patient
populations when conducting a meta-analysis
\cite{Cochrane_Cochrane_2012}. This is a subtle point, and is more
clearly developed in the effect-size meta-analysis realm than in the
meta-analysis of descriptive epidemiological data.  In fact, comparing
studies with different outcome measures is at the heart of this book,
which develops a method for comparing the results of descriptive
epidemiological studies of disease prevalence, incidence, remission,
and mortality risk, which are focused on subpopulations from varying
age groups, sexes, regions, and time periods.

It is not without precedent, however.  The next section discusses the
legacy of ``generic disease modeling'', which my integrative approach
to descriptive epidemiological meta-analysis builds upon.



