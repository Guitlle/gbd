\section{Covariate modeling}

There are two distince points where introducing covariates aids in
disease modeling.  Reducing bias and variance of in-sample fits and
increasing predictive accurace of out-of-sample fits.

Still to write: covariates ``random effects'', macroeconomic derived
covariates, interpreted as shifting the level of the age pattern.
Future work change age pattern itself (``cokreiging'') also relax
linear assumption of shift.  This will be computationally challenging.


\subsection{Covariates to reduce variance}

There is a huge amount of heterogeneity in the input data collected
from the published and unpublished literature, and some of it can be
explained through instrument design, diagnostic criteria, etc. How to
model this TK, including examples.

study level covariates

dummies

numerical covariates

'ignore' column as a special type of covariate

Dealing with ordinal and categorical study level covariates

This is familiar from the STEP model, and only needs to be detailed in
the specific ways that it comes up in disease modeling, e.g. troponin
test in IHD, past year and past month prevalence for substance, etc.

TK covariates to explain dispersion in neg binom model.

\subsection{Covariates to predict out-of-sample}

Data scarcity requires borrowing strength between regions, sexes, and
times. Precise development of each of these pieces, starting with sex
or time, because it is simpler. Visual and numerical examples of how
borrowing between sexes helps things, as well as examples of where it
goes wrong, e.g. if the age pattern is different between men and
women, then borrowing it is a mistake. Emphasize the way this generic
disease model must be appropriately customized for the situation at
hand, for example reproductive health conditions, certain cancers,
different age patterns for different regions.

When predicting disease parameters for regions with little or no
direct measurements of the disease parameters of interest, it can help
to use data on weakly related health, geographic, and macroeconomic TK
as covariates. Explanation of how this is done.  Particular attention
is needed to how it is done at the country level, and then aggregated
up to the regional level. Some evidence (from simulation study?) that
it is a good idea would be nice to present as well.

country-specific values, and how they are aggregated into regional
estimates - fixed reference value

built-in covariates: sex, year, region


