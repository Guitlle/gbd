\section{Covariate modeling}

TK introductoin to covariate modeling, as used in global health, by
way of economertrics and quantitative political science.  This
introduction can quote from and reference: Wikipedia, Gary King's
book, Andrew Gelman's work, some tradition of macroeconomics which I
need to learn more about.

TK distinction and discussion of ``fixed effects'' and ``random
effects'' and how hierarchical modeling blurs these distinctions.

In my integrative model of disease in populations, covariates help
in two distinct and important ways.  First, they help by
explaining the bias and variation of the noisy measurements of
epidemiological rates.  And second, they help by increasing the
validity of out-of-sample fits.


\subsection{Covariates to reduce bias and variance}

As you may recall from the data plots in
Section~\ref{theory-age_group_model-overlapping_data}, there is often
a huge amount of heterogeneity in the input data collected by
systematic review.  Some of this variation can be explained through
covariate modeling.  The age-pattern model in
Section~\ref{theory-age_pattern_model} is a sophisticated example of
this, where I have modeled the complex dependence of rates on age
using Gaussian Processes with Mat\'{e}rn covariance functions and
additional bells  and whistles.  But there are also simple additions
to the rate model that achieve similar improvements.  For example,
including what statisticians might call a slope on study year can
reduce bias and variance when rates are changing slowly over time.
This covariate conflicts philosophically with the stationarity
assumption made in
Section~\ref{theory-forward_sim-compartmental_model-simplying_assumptions},
but it is acceptable in practice as long as the time variation is
small.  It is also quite appropriate if it is the level of diagnosis
that is changing over time, while the rate itself remains constant.
\begin{align*}
r_i n_i &\sim \NegativeBinomial(\mu_i, \delta)\\
\mu_i &= r(a)e^{\beta_0 + \beta_1 y_i}
\end{align*}

There are other relevant covariates that can help to reduce bias and
variance, but they must be selected on a case-by-case basis.  They
fall into two broad categories: study-level covariates and
country-level covariates, although the distinctions can blur.

A prototypical example comes from ischemic heart disease prevalence.
There are a variety of diagnostic tests available and different
studies of IHD prevalence use different diagnostic criteria for case
ascertainment.  Some are more sensitive than others, and this leads to
variation in data with a clear explanation.  By including an indicator
variable as a covariate in each row of data, $x_i = 1$ if row $i$
comes from a study that used a triponen test, and $x_i = 0$ otherwise,
I can fit a model which includes a parameter to ``cross-walk'' between
studies using these two different case ascertainment criteria:
\begin{align*}
r_i n_i &\sim \NegativeBinomial(\mu_i, \delta)\\
\mu_i &= r(a)e^{\beta x_i}
\end{align*}

This same approach can be applied to questions with varying duration
that come up in the meta-analysis of psychological disorders, for
example anxiety disorder is sometimes measures in past month
prevalence and sometimes in past year prevalence.

TK a discussion of the reference value, which will be returned to in
the next section about predictive accuracy.

Another example that leads to the same mathematical model deals with
studies that target specific populations.  For example, systematic
review led to collecting a number of Hepatitis C prevalence studies
that used voluntary blood donors as the sample frame.  This is clearly
not the whole population, but it is not obviously a biased sample.  It
all depends on whether the people who volunteer to give blood have a
higher or lower prevalence of Hep C than the general population.
Disease experts believe that \emph{paid} blood donors are not
representative of the general population, but do not have a strong
prior belief about the representativeness of voluntary donors.  This
is another place where covariate approach is appropriate. In the Hep C
example (to be elaborated on in Chapter~\ref{TK}), the systematic
review coded rows corresponding these voluntary blood donors, as well
as other studies of specific subpopulations, such as mothers visiting
antenatal clinics, with a bias indicator $x_i = 1$, and coded rows
from studies of the general population with $x_i = 0$.  In this case
it is appropriate to introduce a parameter for the bias introduced by
sampling from these subpopulations, but it is also appropriate to
introduce a parameter for these variables to have over-dispersion that
differs from the studies of the general population.

Because fitting the negative binomial model with an uninformative
prior on the over-dispersion term can be tricky, I have ended up
parameterizing the over-dispersion in the general, but slightly opaque manner:
\begin{align*}
r_i n_i &\sim \NegativeBinomial(\mu_i, \delta_i)\\
\mu_i &= r(a)e^{\beta x_i}\\
\log_10(\delta_i-.5) &= \eta + \zeta x_i\\
\eta &\sim N(\mu_\eta, .25^2)\\
\zeta &\sim N(0, .25^2)
\end{align*}

TK figure showing with and without this approach, or possibly a range
of priors on zeta.

In addition to dichotomous variables, like diagnostic criteria or
bias, covariate modeling can be applied to continuous covariates, like
GDP, Animal Fat consumption, PfPR, or anything.  An example which uses
covariates derived from the country where the study took place is in
MS.  Here experts believe that there is a correlation between disease
prevalence and distance from the equator.  By including the absolute
value of latitude (normalized over all countries to have mean 0 and
variance 1), the model is able to reduce the variance slightly.

TK a discussion of the inappropriateness of using this for causal
analysis, as well as a recognition that people are going to be really,
really tempted to.

Additional covariates to consider: conflict for anxiety, PTSD. Health
system access for AF.  Ask Theo what else has been popular.

\subsection{Covariates to predict out-of-sample}

Data scarcity requires borrowing strength between regions, sexes, and
times. Precise development of each of these pieces, starting with sex
or time, because it is simpler. Visual and numerical examples of how
borrowing between sexes helps things, as well as examples of where it
goes wrong, e.g. if the age pattern is different between men and
women, then borrowing it is a mistake. Emphasize the way this generic
disease model must be appropriately customized for the situation at
hand, for example reproductive health conditions, certain cancers,
different age patterns for different regions.

When predicting disease parameters for regions with little or no
direct measurements of the disease parameters of interest, it can help
to use data on weakly related health, geographic, and macroeconomic TK
as covariates. Explanation of how this is done.  Particular attention
is needed to how it is done at the country level, and then aggregated
up to the regional level. Some evidence (from simulation study?) that
it is a good idea would be nice to present as well.

country-specific values, and how they are aggregated into regional
estimates - fixed reference value

built-in covariates: sex, year, region


A covariate for sex or region can be implemented similarly.  For sex,
I coded the categories male, female, and total as $1, -1, 0$ to
capture fact that total is an average of rates for males and females.
For region, I have used a hierarchical random effect, to allow expert
prior on the relationship between regions.

TK description of how regions are clustered into super-regions and how
this hierarchial effect works.  A few examples of how data from one
region affects the other regions.

Year also helps with predictive validity, although it is primarily
important for reducing variation.


\section{Operationalizing the Covariate Modeling}

In the modular approach to implementing the integrative systems model
of disease, the covariate step comes between the age-pattern model and
the age-integrating model.  From the age pattern model, I have a prior
distribution on $\boldmu(a)$, and this is combined with a covariate
model conceptually before, but operationally after being mapped to age
intervals by the age integrating model.  The equations for these
models make this clear.

The covariate model for the mean has the following form:
\[
\boldmu_{r,X_i}(a) = \boldmu(a) e^{\beta X_i},
\]
where $X_i$ is a vector of covariates, and $\beta$ is a vector-valued
covariate effect parameter.

The covariate vector will often include an ``intercept shift'' for
sex, study, country, region, and super-region, as well as a slope on
time and a handful of other relevant covariates about relevant
macroeconomic and epidemiological quantities.

I am now attempting to follow the approach of Girosi and King and not
put priors on these effect coefficients being similar between
different nodes in the hierarchy.  Instead, I will specify priors that
say that the levels and age-patterns for the neighbors in the
hierarchy are assumed to be similar a priori.  This means that if node
$u$ is a child of node $v$ in the hierarchy, and the weight of the
edge between them is $w$, then the prior includes a simliarity
potential term encoding the belief that
\[
\log \boldmu_{u,X_u}(a) - \log \boldmu_{v,X_v}(a) \sim \Normal(0, w^2).
\]

Because I will fit the model in stages for computational efficiency,
this belief has different implications in different phases.  When
generating estimates that are consistent at the level of $u$ or higher
in the hierarchy, I assume that the age pattern $\boldmu_u$ is the
same as the age patterns below it, including $\boldmu_v$.  Therefore,
the similarity priors says only something about the difference
$X_u\beta - X_v\beta$.

When generating estimates that are consistent at the level of $v$, I
\emph{do not} assume that the age pattern for node $v$ is the same as
its parent, but instead I use an ``empirical bayes'' approach, and
take age pattern prior on $\boldmu_{v,X_v}$ to be the previously
estimated value of $\boldmu_{u,X_u}$.  This is combined with all
relevant data, as well as the priors on all rates in the disease model
to obtain a posterior estimate of the level and age pattern for node
$u$.
