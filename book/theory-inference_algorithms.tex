\section{Empirical Bayes approach}
Since ``the model is what you fit, not what you wish you fit'', it is
now appropriate to describe my empirical bayes approach, which
corrupts the beautiful hierarchical model from the previous section
for the practical concern that modern computers be able to produce
reliable estimation relatively quickly.

TK description of the empirical bayes approach, which pools all of the
world's data on each of the individual rates that can be understood
well in isolation (incidence, prevalence, remission, and
excess-mortality), and then applies the regional predictions for each
of these to the compartmental model at the regional level.

A special detailed discussion of the over-dispersion prior and how the
empirical bayes approach affects it.

Two approaches I have considered are the following.  One, to fit an
individual model for each rate function, and use the posterior of each
rate fit in isolation as the empirical prior for a consistent fit at
the region/sex/year level. Or two, to fit a consistent model for all
rates together at the global level, and then use this as the empirical
prior for each region/sex/year.  The former is faster computationally.
The latter seems more sound theoretically.

\section{Computational methods}
TK introduction to numerical computation, a short theory on the
interplay between physics-inspired modeling, statistics-inspired
modeling, and computational realities.

TK history of Bayesian computing, history of MCMC

TK discussion of PyMC as a computational platform

TK discussion of step methods, including a comparison of
AdaptiveMetropolis that continues tuning and AdaptiveMetropolis that
stops tuning at the end of the burn-in period (so that it is an honest
Markov chain).

TK discussion of convergence tests

When MCMC does not converge, there are three approaches I know of: run
the chain for longer, start from better initial values, or use more
appropriate step method. TK Examples of running longer. TK Example of
different initial values, benefits of starting close to the posterior
mode. TK extensive discussion of step methods, and justification for
using the Adaptive Metropolis approach.

TK detailed discussion of empirical Bayes, the computational
efficiency of this approach, the impossibility of using a more
complicated approach.

TK directions for future theoretical research: Bayesian evidence of
mixing; improved bounds for asymptotic running time when the model has
hierarchical structure; hit-and-run sampling; Gibbs and metropolis
sampling; adaptive methods that don't require special analysis to
prove convergence.



\section{A computational benefit of using transformed normal models}

There is a computational benefit to using transformed normal models
(Section TK), which is so important that they should be considered
even if they were not as similar to the theoretically more appealing
models like the negative binomial.  In mathematical terms the
transformed normal models are \emph{differentiable}, and this property
has important implications for how computationally difficult it is to
fit the model.

An analogy to mountain climbing is helpful in understanding how these
models are fit: differentiable functions lead to smooth hills, where
the climber can always tell which direction leads to higher
elveations.  In smooth, \emph{convex} functions, the situation is even
better, because going towards higher elevation always leads the
climber to the top of the mountain eventually.  None of the binomial,
beta-binomial, poisson, or negative binomial models are convex and
none are differentiable.  This means that a mountain climber trying to
follow the steepest ascent may not know where to go, or may get study
at a local maxium this is far from the true mountain top.  Worst of
all, it means that starting the optimization from different intial
values could lead to different estimates.

Transformed normal models avoid the worst of this trouble.  After the
complicatoin to be introduced in following sections, where I will
intergfated of wide age groups, introduce strong priors on age
patters, and connect incidence, prevalence, remission, and mortality
according to the compartmental model of populatoin health, the
posteriors will not necessarily be convex.  But at least they will be
``more differentiable''.  Empirical evidence, to be developed below
agrees wit hthis theory.  Offset log transformed models require less
iterations of MCMC to produce answers within a prescribed tolerance.

TK results of experiment comparing variation in MCMC estimates for
different models.
