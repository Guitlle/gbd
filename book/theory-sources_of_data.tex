\section{Sources of Input Data}

TK discussion of the various sources of input data, and there relative
merits

 Precise definitions of the different input data sources, and
 vignettes about how they could be collected.

The need for careful case definitions and the formal definition of all
data relevant (and some not relevant) to dismod (incidence rate,
prevalence rate, remisson rate, excess-mortality rate, with-condition
mortality rate, cause-specific mortality rate, with-condition
population mortality rate, without-condition mortality rate, all-cause
mortality rate, mortality rate ratio, standardized mortality ratio,
duration)

The production of the regionally representative, age-specific
all-cause mortality data is described in detail in TK. In short TK
summary of the mortality book. TK discussion of resolving the
descrepancies between the age grids. The regional mortality estimates
produced for the global burden of disease study are for TK-year age
intervals, while the age groups in the system dynamics model are
chosen appropriately for the disease being analyzed. To resolve the
possibly incommensurate mesh points that these estimates fall on, we
take the approach of using piecewise constant interpolation to extend
the all-cause mortality rate to a continuous function of age, and then
use the values of this piecewise constant function at the mesh points
of required by the system dynamics model. TK a figure showing this in
action. which I hope we do not have to deal with for GBD because we
have planned everything so precisely, but we may have to deal with in
the future.


PRISMA checklist for systematic review and meta-analysis
