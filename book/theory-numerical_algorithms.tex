\chapter{Numerical Algorithms}

Computational tractability has an important influence on model
development, which often goes unacknowledged. The models that I fit
are often a compromise between models I would like to fit and the
limitations of the algorithms and computing infrastructure
available. This has always been the case, but modern algorithms and
modern computing have shifted the balance tremendously.

In the days before digital computers, computational tractability meant
that models must be simple and computational methods elegant. In the
18th century, for example, an important challenge in predictive
modeling was in navigation \cite{Williams_From_1993}. Forecasting the
path of stars allowed a ship to chart its course accordingly. The
method of least squares, first published at the start of that century
by Legendre, elegantly provided a solution
\cite{Legendre_Nouvelles_2011}\. The method finds an approximate
solution to a set of equations by minimizing the square sum of the
residuals. Using this method, one can plot the location of a celestial
body at different time points, postulate a model form (for instance by
assuming that the body follows a straight line), and then use the
method of least squares to determine the parameters of the model that
best fit the data. Why minimize the square sum of the residuals? Why
not minimize the distance between the data and a proposed solution or
the sum of the distance and the number of parameters in the model? For
one, it has attractive theoretical properties, because it is
equivalent to finding parameters of maximum likelihood given normally
distributed data. But perhaps just as importantly, minimizing the
square sum permits a closed form, analytical solution that was well
matched to the computational resources of the day.

With the development of digital computation, more computationally
intensive methods have become feasible. When Ulam challenged himself
to calculate the probability of winning in solitaire in the 1940s, an
analytic solution was elusive. A simple, but computationally intensive
approximation method was developed. As Ulam later remarked, ``The
question was what are the chances that a Canfield solitaire laid out
with 52 cards will come out successfully? After spending a lot of time
trying to estimate them by pure combinatorial calculations, I wondered
whether a more practical method than `abstract thinking' might not
be to lay it out say one hundred times and simply observe and count
the number of successful plays''. This approach has been
generalized to the Monte Carlo method, a class of computational
methods that rely on repeated random sampling to approximate
calculations that are intractable to calculate exactly
\cite{Eckhardt_Stan_1987}.

It is the successors to the Monte Carlo algorithm that make the
Bayesian methods I use in integrative systems modeling possible.  In
Bayesian terms, the model of process and model of data articulated in
the first half of this book provide a prior distribution and
likelihood.  In equations, it is a simple application of Bayes'
formula to go from this to the posterior distribution.  The exact
computation of the distribution is intractable, however, and it is
algorithms for sampling from the posterior distribution (or a close
approximation thereof) that produce the parameter estimates for my
models.

Bayesian methods were developed contemporaneously to the method of
least squares, but were limited in application before the development
of Markov chain Monte Carlo (MCMC) algorithms and modern computers.
Prior to these innovations, analysis was only tractable for a very
limited class of prior distributions and likelihoods. With sufficient
computing power, the posterior distribution can be sampled from using
Monte Carlo methods instead of being computed analytically
\cite{Gelman_Bayesian_2003}. Monte Carlo methods can also be applied
to integrate the posterior distribution to obtain, for instance, the
posterior mean and variance. As computational resources to apply the
approach to more complex problems have become more widely available,
the approach has gained popularity \cite{Tanner_From_2010}.

My integrative systems model of disease in a population does not admit
a closed form representation for its posterior distribution.  Instead,
I rely on MCMC to draw samples of the model parameters from their
posterior distribution.  Even this requires some care, TK something
about not using Gibbs sampling.  Instead, I use the Metropolis step
method, and the Adaptive Metropolis (AM) variant
\cite{Haario_Adaptive_2001}. The MCMC algorithm often benefits from
initial values when chosen wisely, and this seems to be particularly
true when using MCMC with the AM step method. I use Powell's method
to find initial value for the model parameters for MCMC, and I use
Normal approximation to find initial values for the
variance-covariance matrices in the AM step method.  Furthermore, I
use an \emph{empirical Bayes} approach to separate the global model
into submodels that can be fit in parallel.  The remainder of this
chapter describes each aspect of the numerical algorithm in more
detail.

\section{Markov chain Monte Carlo}
Markov chain Monte Carlo (MCMC) is a class of Monte Carlo methods that
obtains approximate solutions by constructing Markov chains. A Markov
chain is a stochastic process, or a sequence of random variables, such
that the probability distribution of a random variable at one point in
the sequence only depends on the random variable immediately before it
in the sequence. If a Markov chain satisfies certain conditions (TK)
then it must tend towards a unique stationary distribution as the
sequence continues. The key to using the MCMC algorithm for
integrative systems modeling is constructing a Markov chain with the
following three properties:
\begin{enumerate}
\item the stationary distribution of the chain is equal to the
  posterior distribution of the model
\item each step of the chain can be computed efficiently
\item the chain converges to its stationary distribution in a
  reasonable number of steps
\end{enumerate}

A simple example can make this clearer. Example TK, possibly sampling
uniformly from a region in the plane.

\section{Adaptive Metropolis-Hastings MCMC}
TK description of the step method that DisMod 3 uses. To understand
the algorithm, one must first understand the Metropolis-Hastings
algorithm.

In the context of Bayesian statistics, the Metropolis-Hastings
algorithm is a technique used to sample from the posterior
distribution when the posterior distribution cannot be easily sampled
from directly. The algorithm proposes a new direction in the random
walk by sampling from a proposal probability distribution, which
depends on the current location of the walk. TK a version of this that
includes equations.  The proposal is then accepted and the step taken
if a random draw from a uniform distribution between 0 and 1 is less
than the product of the probability of the proposal given by the
target probability distribution and the probability of the proposal
given by the proposal distribution divided the same quantity with the
current state substituted for the proposal state. Otherwise, the
proposal is rejected and a new proposal, and thus a new direction, is
offered. By only needing to compute the ratio of the target
distribution evaluated at the proposal state and the current state, it
vastly simplifies sampling from the target posterior
distribution. This simplification arises, because the normalizing
factor, or the denominator of the posterior distribution (often the
hardest part to calculate), cancels out.

TK example application to sampling from a banana in the plane.

Adaptive Metropolis-Hastings extends Metropolis-Hastings by adaptively
adjusting the variance-covariance matrix for the proposal
distribution, based on the acceptance rate of the proposals.  TK more
correct and mathematical description of this. If few proposals are
getting accepted, then the algorithm increases the variance of the
proposal distribution in order to consider more directions in the
random walk. If a lot of proposals are getting accepted, then the
algorithm decreases the variance of the proposal distribution in order
to zero in on the right direction. DisMod 3 was developed using PyMC,
a Python library that implements the Adaptive Metropolis-Hastings
algorithm as well as many other MCMC methods \cite{Patil_PyMC_2010}.

TK example of sampling from a banana with AM step method.

TK a word about convergence, the theoretical applicability of the AM
method, practical tests of convergence, the need for experimental
testing and validation.  At least 3 general approaches exist for
making a Markov chain's convergence more likely. The first and
simplest approach is to just run the chain for longer and thus take
more samples. The second approach is to use a more appropriate step
method, for example by using Adaptive Metropolis-Hasting over the
Metropolis-Hasting algorithm. Other more complicated step methods can
also be used, but they tend to be less stable. Adaptive
Metropolis-Hasting is one of the more stable, multi-purpose step
methods available.

\section{Initial Values from MAP}
Shorter run time for better initial values. For instance, by
initializing certain parameters to their maximum likelihood values
\cite{Bishop_Neural_1995}.

TK description of powell's method

TK example of fitting a simple model with and without good initial
values.

\section{Empirical Bayes}

This technique falls under the rubric of Empirical Bayes.

Empirical Bayesian approaches estimate the prior distributions in the
model instead of specifying entirely uninformative prior
distributions. In the context of descriptive epidemiology,
appropriately informative prior distributions help guide the model
towards realistic estimates of epidemiological parameters. Without
Empirical prior distributions, the data are often too noisy and sparse
to yield sensible estimates alone or the computational cost of
obtaining sensible estimates would be too large, given the time it
would take for the Markov chain to converge. Only the data combined
with the system dynamics model and an appropriate Empirical prior can
give reliable estimates with current computational constraints.

TK formal definition of what I am doing when I do empirical bayes.

Many challenges remain in fitting complex integrated systems
models. In the early period of regression analysis, the optimization
routines used to find the maximum likelihood estimate of a model's
parameters required constant tweaking and observation in order to
converge. Now, these techniques are incredibly robust and applicable
to a wide range of modeling problems. We are still in a relatively
early stage of fitting complex Bayesian models using Monte Carlo
methods. As computing power increases and these computational methods
progress, modelers will have access to increasingly more reliable and
general purpose fitting algorithms and with that access will come
increasingly realistic and predictive models.

TK example of empirical bayes, comparison to fully Bayesian approach.

\section{Future Directions for Research}

MCMC has been an enabler for this approach.  Without free/libre
open-source software for implementing AM/MCMC this project would not
have been possible.  But it is not the only approach.  Message passing
algorithms have proven themselves TK more on this.  Nonlinear
optimization, is another promising approach TK especially combined
with bootstrap method for estimating uncertainty.  Population monte
carlo
