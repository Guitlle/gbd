This book, \emph{An Integrated Metaregression Framework for Descriptive
  Epidemiology}, is a full-length treatment of new meta-analytic
methods for descriptive epidemiology.  From first principles, it
develops the integrative systems model that constitutes the
theoretical foundation of years lived with disability (YLD) estimation
in burden of disease studies like the Global Burden of Disease, Injuries, and Risk Factors 2010
 (GBD 2010) Study.  The estimation approach relies on producing
age-specific prevalence estimates of the nonfatal outcomes of a vast
array of diseases, injuries, and risk factors.  As part of the GBD
2010 Study, we have developed a Bayesian metaregression tool
specifically for this purpose. This tool estimates a generalized
negative-binomial model for all the epidemiological data with various
types of fixed and random effects.  These include age fixed effects,
fixed effects for covariates that predict country variation in the
quantity of interest, fixed effects that predict variation across
studies due to attributes of the study protocols, and superregion,
region, and country random intercepts.  The tool uses Bayesian
inference of the parameters to sample from the joint posterior
distribution of the model, incorporating all relevant descriptive
epidemiological data.  This approach is new, but the line of research
builds on work in generic disease modeling that has been in use for
almost $20$ years in global health
epidemiology.\cite{Barendregt_Generic_2003} However, until now, the
descriptions of the models and the methods have been scattered throughout
the scientific literature in a loose collection of journal articles,
burden of disease reports, and operations manuals.

This book substantially extends the previous modeling efforts for YLD
estimation in burden of disease estimation by formally connecting a
system dynamics model of disease progression to a statistical model of
the epidemiological rates calculated in descriptive
epidemiological research and collected in a systematic
review.  This combination of systems dynamics modeling and statistical
modeling, which we call
\emph{integrative systems modeling}, allows the model to integrate all
available relevant data.  Because of the advanced numerical algorithms needed to fit these complex models, chapter~\ref{numerical-algorithms} provides the
necessary background on Markov chain Monte Carlo (MCMC) and other
relevant computational methods.

Experience with the results of systematic review indicates that when
all available relevant data are collected, they are often very
\emph{sparse} and very \emph{noisy}.  In GBD estimation, data sparsity
often means that there are whole regions of the globe for which no
data are available.  The sparsity of data means that predictions of prevalence
need to take advantage of relationships to covariates in the
metaregression or default to the average of a region, superregion, or
the world.  Dealing with noisy data is an additional challenge. In the
regions or countries with multiple measurements, the results are often
highly heterogeneous. The degree of heterogeneity is far beyond what
is expected on the basis of sampling error and indicates considerable
nonsampling variance.  The sources of nonsampling variance include
challenges in sample design; lack of a representative sample; and
implementation issues in data collection, case definitions, and
diagnostic technologies.  To make matters more complicated, there
is \emph{true} geographic variation as well.

We will also address a number of other common challenges in estimating the
prevalence of nonfatal outcomes of disease:
\begin{itemize}
\item Based on biology, exposure, or clinical series, we may have strong
priors on the age pattern of incidence or prevalence of a condition;
for example, due to cumulative exposure to carcinogens, we expect the
incidence of many cancers to increase with
age, at least until some
adult age.  Another example is that the prevalence of bipolar
disorder is zero in younger children.

\item Published studies often use nonstandard age groups like $18$--$35$
or $15$ and older.  For the GBD 2010 Study, we needed to use data from different
nonstandard age groups to generate coherent estimates for the $20$
age groups in the study.  Given that prevalence for most sequelae is
strongly related to age, this issue is particularly important.

\item For many conditions, the available studies use
different case definitions.  The review of diabetes prevalence studies
identified $18$ different case definitions in use.  If all non--reference
definition data are excluded, predictions can be based on only a
limited number of studies.  An alternative is to empirically
adjust between different definitions using the overlap
in available studies.

\item Within regions or
countries, the true prevalence for a sequela can vary enormously. The
high level of hepatitis C in Egypt is an example in the Middle East
and North Africa region.  Such within-region heterogeneity in the true
rates must be accommodated in a metaregression framework.

\item Data are collected for many different outcomes,
such as incidence, prevalence, remission, excess mortality, or
cause-specific mortality.  The mix of data varies across diseases and
across regions for a disease.  All these sources provide some
relevant information for estimating prevalence.
\end{itemize}

The statistical model developed in this book focuses
particularly on techniques for handling sparse, noisy data while also
addressing these additional challenges.  The book explores statistical
models for overdispersed count data, covariate modeling to explain
systematic variation in epidemiological data and to increase predictive
accuracy for estimates where no data are available, and age pattern
modeling to systematically incorporate expert knowledge about how
epidemiological rates vary as a function of age.  It also develops a
novel theory of age-group modeling to address the heterogeneity in
age groups that is commonly found during systematic review.

In the first half of this book, we present the theoretical foundations
of integrative systems modeling of disease
in populations.  The second
half of the book contains a series of applications of the model to the
meta-analysis of a dozen different diseases.  These
practical applications demonstrate how the model performs in a variety
of scenarios and also how the model performs when the
model assumptions are violated.

