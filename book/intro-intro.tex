\section{Introduction}
This book, \emph{Integrative systems modeling of disease in
  populations} is a book length treatment of model-based meta-analytic
methods for descriptive epidemiology.  It develops, from first
principles, the integrative systems model which constitutes the
theoretical foundation of Years Lived with Disability (YLD) estimation
in burden of disease studies like the Global Burden of Disease 2010
(GBD2010).  This is an inferential compartmental model of the
progression of disease through a population, and it builds on work in
generic disease modeling that has been in use for over ten years in
global health epidemiology in the popular generic disease modeling
system DisMod II, distributed by the World Health Organization
\ref{DisMod II paper}.  However, until now, the description of the
model and the methods behind that software and its successors have
been scattered through the scientific literature in a loose collection
of journal articles and operations manuals.

In addition to collecting the prior work on compartmental modeling of
disease together in one place, this book significantly extends the
model, by formally connecting the system dynamics model of disease
progression to a statistical model of epidemiological rates, the kind
that are calculated in descriptive epidemiological research and
collected through systematic review.  This combination of systems
dynamics modeling and statistical model, which I call
\emph{integrative systems modeling} allows the model to integrate all
available relevant data.  Because advanced numerical algorithms are
needed to fit these complex models, a section of the book provides the
necessary background on Markov chain Monte Carlo (MCMC) computation.

Experience with the results of systematic review indicates that when
all available relevant data is collected, it is often very
\emph{sparse} and very \emph{noisy}.  The integrative systems models
developed in this book focus particularly on techniques for handling
sparse, noisy data.  The book explores statistical models for
over-dispersed count data, covariate modeling to both explain
systematic variation in epidemiological rate data and increase
predictive accuracy for estimates for subpopulations where no data is
available, and age pattern modeling to systematically incorporate
expert knowledge about how epidemiological rates vary as a function of
age.  It also develops a novel theory of age group modeling to address
heterogeneity in age groups commonly found during systematic review.

The theoretical foundations of integrative systems modeling of disease
in populations consititute the first half of this book, and are
complemented with a series of applications of the model to the
meta-analysis of more than a dozen different diseases.  In the second
half of the book, these practical applications 
demonstrate how the model performs in a variety of scenarios,
and also investigate how the model performs when the model assumptions
are violated, and how this model violations may be addressed.

The book concludes with a detailed description of the future
directions for research in model-based meta-analysis of descriptive
epidemiological data and integrative systems modeling for global
health.
