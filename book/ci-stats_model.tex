\chapter{Statistical Model Computational Infrastructure}

I have implemented statistical model describe above in Python/PyMC,
with Pandas to handle all the data wrangling.  To keep things
manageable, I have broken the model into consistuient elements,
corresponding to the theoretical elements developed in the sections of
Chapter~\ref{TK}.  These are the age-pattern model, the covariate
model, the age-integrating model, the data model, and the consistency
model.


\section{Age-pattern Model}

The age pattern model implements a non-negative, piecewise linear
Gaussian process $\boldmu$ as an exponentiated piecewise-constant spline
with spline effects $\gamma_1,\ldots,\gamma_K$ and knots
$a_1,\ldots,a_K$.  The \Matern covariance function is included by
giving $(\gamma_k)_{k=1}^K$ a multivariate Normal prior distribution
with the appropriate variance-covariance function (represented as a
Cholesky factored $k\times k$ matrix for computational efficiency).

This should be checked to confirm that adding knots does not change
the level of smoothing, which can be done theoretically or
experimentally.


\section{Age-integrating Model}

The age-integrating model maps the age pattern to age intervals,
according to the data.  There are several approach to this that I
would like to explore, and the simplest is to approximate the integral
of the age pattern from $a_0$ to $a_1$ by the midpoint,
\[
\mu_i = \int_{a_{0,i}}^_{a_{1,i}} \boldmu(a) d\boldw_i(a) \approx \boldmu\left(\frac{a_0+a_1}{2}\right).
\]

I plan to demonstrate that this is not appropriate for much of the
data we deal with, but it is a good starting place to make sure
everything works together, and it will provide a way to assess the
cost/accuracy tradeoff of more precise approximations to the integral.


\section{Covariate Model}

The covariate model is operationalized after the age-integrating model
as a matter of computational efficiency.  Assuming that the covariates
in the model do not interact with age, this is mathematically
equivalent to incorporating the covariate before integration, but it
may need to be revisited if it becomes important to have covariates
acting differently at different ages.

Once the ``design matrix'' is buit, the covariate model is quite simple,
\[
\pi_i = \mu_i e^{X_i\beta}.
\]

There is some work that goes into getting the $X$ matrix all correct,
though.  It does not need an ``offset'' column, because the
age-pattern model handles that with $\bar{gamma}$.  Thus it is
convenient to normalize continuous columns of $X$ to have mean zero
and variance one.  Because sex and time are such important factors in
descriptive epidemiology, $X$ always includes columns for these
covariates.  Sex is coded as a value of 0 or 1, while time is shifted
to be 0 at year 2000.

The geographic areas are also included in the covariates, in a
hierarchical formulation taken from the model specification.

Since the data to inform the time, sex, and area covariate effects is
very sparse and noisy, they are modeled as ``random effects'', which
is to say they effect coefficients are given priors of the form
\[
\beta_i \sim \Normal(0, \sigma_{\beta_i}^2),
\]
where each $\sigma_{\beta_i}$ is a parameter with a diffuse hyper-prior
\[
\sigma_{\beta_i} \sim \Gamma(.1, .1).
\]
Section~\ref{TK} explores the effects of this choice of hyper-prior.


On top of this, macroeconomic and demographic covariates are included
as columns based on expert judgement as well as the data-driven
approch of seeing what covariates are important for predicting
population level mortality rates with the CODE model. In the same
manner, study-specific covariates are also included, for example, to
indicate that the study considered a subpopulation that might not be
representative, or used a non-standard diagnostic critera.

The effect coefficients for these columns are modeled as ``fixed
effects'', which means that they are given uninformative priors to let
the data drive the results.

All columns in the input data prefixed with an ``x'' are used in the
$X$ matrix.  These columns must also appear in the output template.

This same covariate modeling approach is also used to predict the
dispersion term for the negative binomial model.  All columns in the
input data prefixed with a ``u'' are used in the $U$ matrix for
predicting $\delta_i$, according to
\[
\delta_i = e^{\eta + U_i\zeta},
\]
where $\eta$ has a somewhat complicated distribution, described in
Section~\ref{TK}, to avoid numerical instability.
