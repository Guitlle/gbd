\chapter{Statistical Model Computational Infrastructure}

I have implemented statistical model describe above in Python/PyMC,
with Pandas to handle all the data wrangling.  To keep things
manageable, I have broken the model into consistuient elements,
corresponding to the theoretical elements developed in the sections of
Chapter~\ref{TK}.  These are the age-pattern model, the covariate
model, the age-integrating model, the data model, and the consistency
model.


\section{Age-pattern Model}

The age pattern model implements a non-negative, piecewise linear
Gaussian process $\boldmu$ as an exponentiated piecewise-constant spline
with spline effects $\gamma_1,\ldots,\gamma_K$ and knots
$a_1,\ldots,a_K$.  The \Matern covariance function is included by
giving $(\gamma_k)_{k=1}^K$ a multivariate Normal prior distribution
with the appropriate variance-covariance function (represented as a
Cholesky factored $k\times k$ matrix for computational efficiency).

This should be checked to confirm that adding knots does not change
the level of smoothing, which can be done theoretically or
experimentally.


\section{Age-integrating Model}

The age-integrating model maps the age pattern to age intervals,
according to the data.  There are several approach to this that I
would like to explore, and the simplest is to approximate the integral
of the age pattern from $a_0$ to $a_1$ by the midpoint,
\[
\mu_i = \int_{a_{0,i}}^_{a_{1,i}} \boldmu(a) d\boldw_i(a) \approx \boldmu\left(\frac{a_0+a_1}{2}\right).
\]

I plan to demonstrate that this is not appropriate for much of the
data we deal with, but it is a good starting place to make sure
everything works together, and it will provide a way to assess the
cost/accuracy tradeoff of more precise approximations to the integral.


\section{Covariate Model}

The covariate model is operationalized after the age-integrating model
as a matter of computational efficiency.  Assuming that the covariates
in the model do not interact with age, this is mathematically
equivalent to incorporating the covariate before integration, but it
may need to be revisited if it becomes important to have covariates
acting differently at different ages.

Once the ``design matrix'' is buit, the covariate model is quite simple,
\[
\pi_i = \mu_i e^{U_i\alpha_i + X_i\beta}.
\]

There is some work that goes into getting the $U$ and $X$ matrices all
correct, though.  It does not need an ``offset'' column, because the
age-pattern model handles that with $\bar{gamma}$.  Thus it is
convenient to normalize some columns of $X$ to have mean zero
and variance one.  Because sex and time are such important factors in
descriptive epidemiology, $U$ always includes columns for these
covariates.  Sex is coded as a value of 0 or 1, while time is shifted
to be 0 at year 2000.

The geographic areas are also included in the $U$ matrix, in a
hierarchical formulation taken from the model specification.

Since the data to inform the time, sex, and area covariate effects is
very sparse and noisy, they are modeled as ``random effects'' (hence
the $U$ matrix and $\alpha$ parameter being distinct from the $X$ and
$\beta$).  To be precise, this means that the effect coefficients are given priors of
the form
\[
\alpha_i \sim \Normal(0, \sigma_{\alpha_i}^2),
\]
where each $\sigma_{\alpha_i}$ is a parameter with a diffuse hyper-prior
\[
\sigma_{\alpha_i} \sim \Gamma(.1, .1).
\]
Section~\ref{TK} explores the effects of this choice of hyper-prior.


On top of this, macroeconomic and demographic covariates are included
as columns based on expert judgement as well as the data-driven
approch of seeing what covariates are important for predicting
population level mortality rates with the CODE model. In the same
manner, study-specific covariates are also included, for example, to
indicate that the study considered a subpopulation that might not be
representative, or used a non-standard diagnostic critera.

The effect coefficients for these columns are modeled as ``fixed
effects'', which means that they are given uninformative priors to let
the data drive the results.

All columns in the input data prefixed with an ``x'' are used in the
$X$ matrix.  These columns must also appear in the output template.

This same covariate modeling approach is also used to predict the
dispersion term for the negative binomial model.  All columns in the
input data prefixed with a ``u'' are used in the $U$ matrix for
predicting $\delta_i$, according to
\[
\delta_i = e^{\eta + U_i\zeta},
\]
where $\eta$ has a somewhat complicated distribution, described in
Section~\ref{TK}, to avoid numerical instability.


\section{Hierarchical Similarity Priors}
Similarity between levels and age patterns of data models from
neighboring nodes in the hierarchy graph are implemented through a
similarity potential, using the expert derived similarity weights.  I
wrote this up already somewhere, so I will need to find those words
and copy them to here.

The important details are in the implementation, which is through a
multistage approach, inspired by empirical Bayesian methods.  The
method begins by generating a fit for the nodes highest in the areas
hierarchy.  The fit at this level is used as a prior on the rate
pattern for the nodes below it in the hierarchy, through the following
potential
\[
\log \boldmu_{n_p}(a) - \log \boldmu_{n_c}(a) \sim \Normal(0,
w_{p,c}^2)
\]

The tricky thing here is determining what to smooth.  Is it really
$\boldmu$ that should be similar, or is it $\boldpi$?  The answer is
relevant if the covariate effects are allowed to vary for different
nodes in the hierarchy.  It is really predictions for the region,
based on the covariates that are believed to be similar (I think), so
it should be $\boldpi_{n_p}$ and $\boldpi_{n_c}$ that are similar.
This requires generating estimates of $\boldpi_{n_c}$ as part of the
model, which either requires aggregating the estimates for the child
node's children using the covariates in the output template, or else
requires making estimates for the child node directly, which means
that it should be in the output template itself, or at least needs to
be derived from the output template.

I added a ``noise floor'' to this as well, I'm not sure if it will be
useful.  This is a way to deal with zeros in the predicted rate, by
saying that nothing smaller than a specified value should be
considered.

\section{Prediction}
The age-specific rate for each node in the hierarchy is something that
could be helpful to have at each step of the MCMC.  To approach it
honestly, I begin with the formulation
\[
\boldpi_{p}(a) = \sum_{c: p\rightarrow c} w_c \boldpi_c(a) / \sum_{c}
w_c,
\]
where $w_c$ is the population weight for child $c$.

Since the model for stage $i$ based on node $n$ assumes that the age pattern is the same
for all descendents of node $n$, this simplifies to
\[
\boldpi_p(a) = \boldpi_n(a) \sum_c w_c e^{X_c\beta} / \sum_c w_c,
\]
which can be computed recursively by a depth first transveral of the graph.

I am not up for this right now, so I will come back to it.

\section{Consistency}
The compartmental model gives a system of differential equations for
the relationship between incidence, remission, mortality, and
prevalence, and this is implemented in tandem with the statistical
model described above, using the similarity prior described above, by
saying that it would be very surprising to find that the prevalence
deviates from the solution to the differential equations by more than
0.1\%.


