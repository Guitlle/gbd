\chapter{Adjusting Model Data}
Since the model data is stored in human-readable files, it is possible
to edit all model parameters with a text editor.  This can be
inconvenient, however, and is prone to error, so DM3 includes a
collection of scripts to make common adjustments to the model data.

\section{Principles of Modeling}
The triumph and the pitfall of Bayesian modeling is that it is based
on a foundation of subjective probability.  While this is appealing in
its logical consistency and its ability to integrate sparse, noisy,
dissaperate data, it also has a problem.  Specifically, it has a
problem in credibility.  Great care must be taken to make sure that
posteriors reflect the ``data'' appropriately and are more than a
complicated way to justify the analyst's ``priors''.

These principles of modeling help guard against subjectivity in
integrative modeling of descriptive epidemiological data.  They are:
\begin{itemize}
\item Start simple, and add complexity gradually
\item Justify all assumptions
\item Keep thorough, concise notes
\end{itemize}

These principles are best demonstrated through example, and the second
half of the DisMod book contains many demonstrative examples.  Here, I
have collected descriptions of the tools and computational
infrastructure which can help you adhear to these principles.


\subsection{Thorough notes and revision control}
The dream of the ``Open Notebook'' is quite alluring, and is the ideal
realization of the third principle of thorough, concise notes.  DM3
can facilitate a workflow that provides a lightweight approach to the
open notebook through the git revision control system.

This works by simply making the directory which holds the Model Data a
git repository, and then by streamlining git commits when using the
model-data-adjusting scripts.

In exchange for almost no overhead, DM3 can store a complete log of
the development of the model, and allow the analyst to compare current
results to results from any other model setting that has already been
run.


\subsection{Starting simple}
DM3 has the capacity for a large amount of complexity, which is why it
is important to start with a simple model (even a model that is too
simple), and add complexity step-by-step.  This will allow analysts
and their discontents to understand the effects of all model
assumptions.

For example, a model for a single geographic area is simpler to
understand than a multilevel hierarchical model.  I therefore
recommend starting a DisMod analysis by focusing on a single region.
It is also good to start with a moderately sized, moderately
consistent data set.  It is much harder to understand the effect of
model assumptions when the estimates must be compared to very sparse
and noisy data.

Another part of starting simple is to begin with a model with few
``effective parameters''.  This is accomplished by using a coarse age
mesh and strong priors on the similarity between geographic areas,
times, ages, and sexes.  Starting with this is analogous to ``full
pooling'' in a linear regression setting, and provides an important
point of comparison if and when the similarity assumptions are
relaxed.


\subsection{Justification of assumptions}
Justifying assumptions works hand-and-hand with thorough notes, and
describing the reason for, and expected effects of, each incremental
change to the model will pay of handsomely when writing up your
results.

There are two parts to this.  First of all, it demonstrates that the
modeling choices made by the analyst all have a reason, and are not
merely subjective choices to make the results match some pre-conceived
notion that is not supported by data.  Secondly, it is often required
to include an analysis of sensitivity and uncertainty in a global
health publication in this modern age, and the experiences in the
model development phase of research offer important insights about
which parameters the estimates are sensitive to.

\section{Tools}
\subsection{Input Data CSV Validator}
It is possible to edit the input data CSV in a text editor, and it is
possible to load it into excel and edit it with a user-friendly
spreadsheet interface.  But it is important to confirm that all of the
input data makes sense, and text editors and spreadsheet applications
are not as helpful with that as a customized tool.

The input data csv validator script addresses this, by reading through
the input data csv, and listing all entries which are wrong or
potentially wrong.

TK example of input data csv validator catching a few errors, and
giving a few warnings.

\subsection{Age-weight merging tool}
TK description of the script to merge age weights into the input data
and output template files.

\subsection{Covariate merging tool}
TK description of the script to merge covariate data into the input
data and output template files.

\subsection{Prior adjustment}
Changing priors with a text editor, and a tool to validate the changes
that also does a git commit automatically.

\subsection{Adjusting the hierarchy}
A tool to validate changes to the hierarchy, and confirm that
the output template contains the needed information for producing
predictions at all nodes at the lowest level of the hierarchy.

\subsection{Adjusting the nodes to fit list}
A tool to validate the nodes to fit list, and to produce a description
of the plan for fitting the model, i.e. a text description of how to
borrow strength between geographic areas.


